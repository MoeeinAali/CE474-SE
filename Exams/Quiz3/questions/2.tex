\subsubsection*{الف}
\subsubsection*{Abstraction}
انتزاع به معنای تمرکز بر ویژگی‌های ضروری یک شیء یا مفهوم است، بدون توجه به جزئیات غیرضروری. این مفهوم کمک می‌کند تا پیچیدگی کاهش یابد و طراحی ساده‌تر شود.
انتزاع مشخص می‌کند که یک سیستم یا مؤلفه باید چه کاری انجام دهد، نه اینکه چگونه آن را انجام می‌دهد.
سیستم به چندین سطح انتزاع تقسیم می‌شود، از کلی‌ترین سطح تا سطوح دقیق‌تر.
مثلا در کلاس Car می‌توانیم متدهای عمومی مانند drive و start و stop را داشته باشیم اما جزئیات پیاده‌سازی این متدها (مانند نحوه کار کردن موتور) برای کاربر مخفی باشد و او فقط با عملکرد کلی این متدها آشنا شود.

\subsubsection*{Information-Hiding}
پنهان‌سازی اطلاعات به معنای محدود کردن دسترسی به جزئیات داخلی یک مؤلفه یا کلاس است تا کاربران یا بخش‌های دیگر سیستم تنها بتوانند از طریق رابط‌های مشخص و تعریف‌شده با آن تعامل داشته باشند. این مفهوم با استفاده از اصول Encapsulation در طراحی پیاده‌سازی می‌شود. پنهان‌سازی اطلاعات بر نحوه دسترسی و استفاده از داده‌ها و متدها تمرکز دارد. جزئیات داخلی مانند متغیرها و متدهای خصوصی از دید کاربر یا کلاس‌های دیگر مخفی می‌شوند. تغییر در جزئیات پیاده‌سازی تأثیری بر سایر بخش‌های سیستم ندارد، زیرا آن‌ها تنها با رابط Interface تعامل دارند.
\subsubsection*{ب}

Abstraction و Information-Hiding با همکاری یکدیگر کیفیت نرم‌افزار را از طریق کاهش پیچیدگی، افزایش قابلیت نگهداری، گسترش‌پذیری و استفاده مجدد بهبود می‌بخشند. انتزاع با تمرکز بر چیستی سیستم، تنها مفاهیم اصلی و رابط‌های مورد نیاز را نمایش می‌دهد، در حالی که پنهان‌سازی اطلاعات با محدود کردن دسترسی به جزئیات داخلی، از وابستگی و اختلال در بخش‌های دیگر جلوگیری می‌کند. این همکاری امکان مدیریت تغییرات، تست آسان‌تر، و گسترش سیستم بدون تأثیر بر سایر مؤلفه‌ها را فراهم کرده و نرم‌افزاری انعطاف‌پذیر و پایدار ایجاد می‌کند.