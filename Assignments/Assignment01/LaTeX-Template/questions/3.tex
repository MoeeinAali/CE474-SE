\subsectionaddtolist{بخش اول}
در روش اسکرام، یکی از اصول اصلی این است که برنامه‌ریزی اسپرینت، به‌طور مشخص برای یک بازه زمانی کوتاه‌مدت انجام می‌شود و پس از شروع اسپرینت، از تغییر برنامه و افزودن یا حذف آیتم‌ها تا حد ممکن خودداری می‌شود. این اصل به تیم کمک می‌کند که تمرکز خود را حفظ کرده و بتواند به اهداف اسپرینت پایبند بماند. با این حال، برخی شرایط ممکن است نیاز به بازنگری در برنامه‌ریزی را ضروری کند.
با در نظر گرفتن شرایط پروژه و ویژگی‌های خاصی که به‌روزرسانی‌های مداوم نیازمندی‌ها ایجاد می‌کند، بهترین رویکرد این است که برنامه‌ریزی اسپرینت فعلی را حفظ کنیم و تغییرات جدید را به اسپرینت‌های آینده موکول کنیم. دلیل این تصمیم:

\begin{itemize}
    \item حفظ تمرکز بر اهداف فعلی: تیم تنها سه روز از اسپرینت اول را پشت سر گذاشته و در حال حاضر اهداف و کارهای خاصی را برای این دوره تعریف کرده است. تغییر در برنامه اسپرینت در این مرحله ممکن است تمرکز و پیشرفت تیم را مختل کند و باعث کاهش بهره‌وری شود.
    \item اولویت‌دهی و ارزیابی تغییرات جدید در برنامه‌ریزی بعدی: طبق اصول اسکرام، تیم در انتهای هر اسپرینت جلسه بازنگری و برنامه‌ریزی برای اسپرینت بعدی برگزار می‌کند. با استفاده از این روش، می‌توان به جای ایجاد تغییرات در اسپرینت جاری، قابلیت‌های جدید معرفی‌شده را در جلسه برنامه‌ریزی اسپرینت بعدی بررسی و اولویت‌بندی کرد و در صورت نیاز، آن‌ها را به فهرست کارهای اسپرینت بعد اضافه کرد.
    \item توجه به محدودیت زمانی پروژه: با توجه به زمان‌بندی فشرده شش‌ماهه پروژه، تیم باید به‌طور مداوم به اولویت‌بندی نیازمندی‌ها و ویژگی‌ها بپردازد تا تمرکز خود را بر روی وظایف اصلی حفظ کند. انجام تغییرات ناگهانی و غیرضروری در اسپرینت‌های جاری می‌تواند به طولانی شدن پروژه و عدم رعایت مهلت مقرر منجر شود.
\end{itemize}

بنابراین، با حفظ برنامه‌ریزی اسپرینت فعلی و موکول کردن ارزیابی تغییرات به اسپرینت بعدی، تیم می‌تواند به‌صورت مؤثر و در عین حال انعطاف‌پذیر، قابلیت‌های جدید را در پروژه ادغام کند.

\subsectionaddtolist{بخش دوم}

با توجه به شرایط همکاری با سازمان ملی اسناد و طبیعت پویای پروژه‌ای که نیازمند تطبیق مداوم با نوآوری‌ها و ویژگی‌های جدید در زمینه مدل‌های زبانی است، روش چابک و به‌ویژه اسکرام به عنوان روشگانی مناسب برای مدیریت این پروژه پیشنهاد می‌شود. در ادامه دلایل انتخاب این روش را با توجه به ویژگی‌های خاص پروژه توضیح می‌دهم:

\begin{itemize}
    \item انعطاف‌پذیری در برابر تغییرات پویا و غیرقابل پیش‌بینی: 
    
    در این پروژه، نیازمندی‌ها ممکن است به دلیل تغییرات سریع در فناوری مدل‌های زبانی و رقابت شدید در این حوزه به‌طور مداوم تغییر کنند. از آنجا که سما انتظار دارد تمامی قابلیت‌ها و نوآوری‌های رقبا، به‌ویژه قابلیت‌هایی که توسط شرکت‌هایی مانند گوگل و غیره معرفی می‌شوند، پیش از انتشار در محصول خود اعمال شود، لازم است که تیم توسعه توانایی واکنش سریع به تغییرات و تطبیق با شرایط جدید را داشته باشد. روش اسکرام، با تقسیم پروژه به اسپرینت‌های کوتاه و بازبینی نیازمندی‌ها در پایان هر اسپرینت، این انعطاف‌پذیری را فراهم می‌کند و اجازه می‌دهد تا تیم به‌راحتی تغییرات و ویژگی‌های جدید را در برنامه‌ریزی اسپرینت‌های بعدی مدنظر قرار دهد.
    \item برگزاری جلسات بازنگری و بهبود مستمر: 
    
    همکاری نزدیک و هماهنگی با سما به دلیل ماهیت دولتی و نظارت دقیق این نهاد، نیازمند شفافیت و ارزیابی مداوم است. در روش اسکرام، جلسات اسپرینت ریویو (Sprint Review) و رتروسپکتیو (Retrospective) پس از هر اسپرینت به تیم امکان می‌دهند تا به طور مستمر عملکرد خود را ارزیابی کرده و بر اساس بازخوردهای سما یا نیازهای جدید، مسیر پروژه را بهبود بخشند. این جلسات می‌توانند به هماهنگی بهتر بین تیم و سما کمک کرده و شفافیت لازم را برای مدیران این نهاد فراهم کنند.
\end{itemize}

\subsectionaddtolist{بخش سوم}

این اقدامات در دو حوزه اصلی یعنی تست و ارزیابی کیفیت و استقرار و آماده‌سازی نهایی محصول قرار می‌گیرند. در ادامه به دو اصل کلیدی برای هر بخش اشاره می‌کنم.

\subsubsection*{تست و ارزیابی کیفیت}
\begin{itemize}
    \item
    اصل تست جامع و چندمرحله‌ای:
    
    یکی از اصول کلیدی پس از کدنویسی، انجام تست‌های جامع است. این شامل تست واحد، تست یکپارچه‌سازی، و تست سیستم است. در تست واحد، هر ماژول و تابع به‌صورت جداگانه مورد بررسی قرار می‌گیرد تا اطمینان حاصل شود که به‌درستی کار می‌کند. در تست یکپارچه‌سازی، عملکرد بخش‌های مختلف سیستم در کنار یکدیگر ارزیابی می‌شود تا اطمینان حاصل شود که ارتباطات و تعاملات بین آن‌ها بدون مشکل است. در نهایت، در تست سیستم، محصول به‌صورت کلی و از دیدگاه کاربر ارزیابی می‌شود.
    
    \item
    اصل تست پذیرش و تطابق با نیازمندی‌های مشتری: 
    
    پس از انجام تست‌های داخلی، لازم است تست پذیرش انجام شود تا مطابقت محصول با نیازمندی‌های تعریف‌ شده مشتری ارزیابی شود. در این مرحله، محصول به‌صورت آزمایشی و با سناریوهای واقعی شبیه‌سازی‌شده بررسی می‌شود. این مرحله کمک می‌کند تا اطمینان حاصل شود که محصول نهایی نیازهای خاص سما را برآورده می‌کند، از جمله امنیت داده‌ها، قابلیت‌های مورد انتظار و همچنین تطابق با مقررات خاص.
\end{itemize}

\subsubsection*{استقرار و آماده‌سازی نهایی محصول}
\begin{itemize}
    \item
    اصل تهیه مستندات جامع و آموزش کاربران:

    پیش از تحویل محصول به مشتری، تهیه مستندات جامع برای کاربران و تیم فنی سما ضروری است. این مستندات شامل راهنمای کاربری، آموزش نصب و راه‌اندازی، و توضیحات فنی مربوط به عملکرد و ویژگی‌های سیستم است. با ارائه این مستندات و برگزاری جلسات آموزشی، کاربران نهایی و تیم فنی می‌توانند به‌راحتی با سیستم کار کنند و از آن بهره‌برداری کنند.
    \item اصل آماده‌سازی زیرساخت‌های پشتیبانی و خدمات پس از فروش:
    
    یکی از اصول مهم پس از اتمام کدنویسی و پیش از ارائه محصول، فراهم کردن زیرساخت‌های پشتیبانی و خدمات پس از فروش است. این شامل آماده‌سازی تیم پشتیبانی، تعیین کانال‌های ارتباطی برای گزارش مشکلات، و اطمینان از داشتن ابزارهای لازم برای نظارت و رفع اشکالات احتمالی می‌باشد. این مرحله به سازمان اطمینان می‌دهد که در صورت بروز مشکلات پس از استقرار، تیم پشتیبانی آماده پاسخگویی به مشکلات و رفع آن‌ها است.
\end{itemize}

\pagebreak

\subsectionaddtolist{بخش چهارم}

قبل از افزودن این قابلیت به بک‌لاگ، مراحل زیر را برای ارزیابی دقیق و تصمیم‌گیری مناسب انجام دهم:

\begin{enumerate}
\item ارزیابی و تحلیل قابلیت پیشنهادی

\begin{itemize}
    \item بررسی نیاز و ارزش افزوده: ابتدا باید مشخص کنم که این قابلیت پیشنهادی چه ارزش افزوده‌ای برای پروژه به ارمغان می‌آورد و چگونه به اهداف اصلی پروژه کمک می‌کند. از توسعه‌دهنده می‌خواهم توضیحات دقیقی درباره مزایای این قابلیت، تأثیر آن بر تجربه کاربران و نحوه تطبیق آن با نیازهای نهایی سازمان ملی اسناد (سما) ارائه دهد. بررسی این موارد به من کمک می‌کند تا تعیین کنم آیا این قابلیت به افزایش کیفیت و جذابیت محصول نهایی کمک می‌کند یا خیر.

    \item تحلیل اولویت و تطابق با زمان‌بندی: با توجه به محدودیت زمانی پروژه (بازه شش‌ماهه)، باید مشخص شود که افزودن این قابلیت چگونه بر برنامه کلی پروژه و اسپرینت‌های آینده تأثیر می‌گذارد. از آنجا که پروژه تحت چارچوب اسکرام مدیریت می‌شود، باید تعیین کنم که این قابلیت چقدر اولویت دارد و آیا می‌توان آن را در اسپرینت‌های آینده جای داد یا خیر. در صورتی که قابلیت پیشنهادی نیاز به تغییر برنامه‌های فعلی داشته باشد، ممکن است افزودن آن به بک‌لاگ به تأخیر بیفتد.
\end{itemize}

\item برگزاری جلسه با تیم و مدیر پروژه

\begin{itemize}
    \item مشارکت تیم در تصمیم‌گیری: با برگزاری یک جلسه کوتاه با اعضای تیم، نظرات و دیدگاه‌های آن‌ها را در مورد قابلیت پیشنهادی جمع‌آوری می‌کنم. از آنجا که توسعه‌دهندگان دیگر نیز در این پروژه نقش دارند، نظرات آن‌ها می‌تواند به شناسایی چالش‌ها و فرصت‌های احتمالی کمک کند و به تصمیم‌گیری دقیق‌تر کمک کند.

    \item مشورت با مدیر پروژه یا مالک محصول: تصمیم نهایی درباره افزودن قابلیت جدید باید با هماهنگی مالک محصول یا مدیر پروژه گرفته شود. مالک محصول می‌تواند در تعیین اولویت این قابلیت نسبت به سایر آیتم‌های بک‌لاگ و ارتباط آن با اهداف کلی پروژه نقش کلیدی داشته باشد. اگر قابلیت پیشنهادی با اهداف و نیازهای اصلی پروژه سازگار باشد، آن را با اولویت مناسب به بک‌لاگ اضافه می‌کنیم.
\end{itemize}


\item افزودن قابلیت به بک‌لاگ با اولویت‌بندی مناسب

\begin{itemize}
    \item پس از ارزیابی دقیق و هماهنگی با تیم و مدیر پروژه، در صورت تایید، این قابلیت به بک‌لاگ پروژه اضافه می‌شود. سپس آن را بر اساس اولویت و تأثیر آن بر پروژه دسته‌بندی کرده و زمان‌بندی مناسبی برای اجرای آن تعیین می‌کنم. اگر این قابلیت از اولویت بالایی برخوردار باشد، می‌توان آن را در اسپرینت‌های آینده در نظر گرفت.

\end{itemize}



\end{enumerate}

\subsectionaddtolist{بخش پنجم}

\begin{enumerate}
    \item عدم تمرکز و طولانی بودن جلسه بازبینی
    
    جلسه بازبینی اسپرینت طبق اصول اسکرام باید کوتاه، منظم و متمرکز بر بازخورد از اسپرینت اخیر و بررسی محصول باشد. با توجه به توضیحات ارائه‌شده، جلسه بازبینی به حدود پنج ساعت کشیده شده و شامل بحث‌های پراکنده‌ای بوده که بیشتر به مشکلات ابزارها و فرآیندهای مدیریتی پرداخته است. این امر نشان می‌دهد که تمرکز جلسه از بازبینی دستاوردهای اسپرینت و ارزیابی محصول منحرف شده و به مسائل جانبی پرداخته شده است. جلسه بازبینی اسپرینت باید بر بازخورد از محصول و آماده‌سازی برای بهبودهای فنی و عملکردی در آینده تمرکز کند و مسائل تیمی و فرآیندهای کاری بهتر است در جلسه بازنگری بررسی شود.
   
    
    \item نیاز به مدیریت فشار کاری تیم

    موضوع دیگری که در جلسه مطرح شده، گله‌مندی تیم از درخواست اضافه‌کاری‌های زیاد به دلیل مشکلاتی مانند مریضی یک عضو و نزدیکی به ددلاین بوده است. این امر نشان می‌دهد که توزیع وظایف و برنامه‌ریزی به‌گونه‌ای انجام نشده که تیم بتواند بدون نیاز به فشار کاری زیاد به اهداف اسپرینت دست یابد. اسکرام به شدت به اهمیت پایداری و تعادل در کار تاکید دارد تا از خستگی و فرسودگی تیم جلوگیری شود. مالک محصول و اسکرام مستر باید به جای درخواست اضافه‌کاری از اعضای تیم، به بررسی مجدد بک‌لاگ و اولویت‌بندی بهتر وظایف بپردازند و در صورت نیاز، مقیاس یا دامنه کاری را تنظیم کنند. این رویکرد می‌تواند به بهبود انگیزه و بهره‌وری تیم کمک کند و فشار کاری را کاهش دهد.
    
    \item تداخل میان وظایف اسکرام مستر و مالک محصول
    
    از آنجا که مالک محصول در این اسپرینت درخواست اضافه‌کاری زیادی داشته، به نظر می‌رسد که تمرکز بیش از حد روی تحویل سریع به جای پایداری و سلامت کاری تیم بوده است. یکی از وظایف اسکرام مستر، حمایت از تیم در برابر درخواست‌های خارج از توان و اطمینان از رعایت اصول چابک در پروژه است. این رویکرد برای حفظ تعادل و سلامت تیم بسیار حیاتی است.
    \end{enumerate}