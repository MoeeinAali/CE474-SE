\subsectionaddtolist{بخش اول}
با توجه به اسلایدها، 
سلسله اقدامات پیشنهادی از آغاز تا تحویل محصول را به صورت گام به گام شرح می‌دهم
:


\subsubsection*{1. ارتباطات}
\begin{itemize}
    \item
    اقدام اولیه: اولین گام درک عمیق نیازها و خواسته‌های شرکت است. برای این منظور، جلسه‌ای با ذینفعان اصلی پروژه برگزار می‌شود تا دیدی واضح از چشم‌انداز پروژه و اهداف اصلی آنها در پیاده‌سازی این فناوری به دست آوریم.
    \item 
    تعریف ذینفعان: شناسایی افرادی که در پروژه نقشی دارند و بررسی دقیق نیازها، محدودیت‌ها و انتظارات آنان، که شامل مدیران، تیم فنی و بازاریابی شرکت فرش گستران سنت می‌شود.
    \item   
    گردآوری اطلاعات نامشخص: در این مرحله، جزئیاتی مانند مشخصات دقیق تکنولوژی‌های مورد نظر، انتظارات از محصول نهایی، و محدودیت‌های فنی و بودجه‌ای بررسی می‌شود.
    \item 
    نمایش گرافیکی مشکل: اگر امکان‌پذیر باشد، می‌توان از نمودارها و مدل‌های گرافیکی برای نمایش بهتر نیازمندی‌ها و الزامات اولیه استفاده کرد.
\end{itemize}

\subsubsection*{2. مدل‌سازی و طراحی اولیه}

\begin{itemize}
    \item 
    طرح یک راه‌حل: با استفاده از اطلاعات به‌دست آمده، قدم بعدی تدوین یک برنامه‌ریزی دقیق است. در این مرحله طراحی مدل اولیه صورت می‌گیرد که شامل تعیین معماری نرم‌افزار، ابزارهای هوش مصنوعی مورد استفاده، و سایر فناوری‌ها است.
    \item 
    بررسی موارد مشابه: تجزیه و تحلیل نمونه‌های موفق مشابه، مانند استفاده از هوش مصنوعی در سایر صنایع، به تعیین الگوها و روش‌های مفید کمک می‌کند.
    \item 
    بررسی راه‌حل‌های موجود: بررسی ابزارها و نرم‌افزارهای موجود که می‌توانند در این پروژه مورد استفاده قرار گیرند و در صورت نیاز به سفارشی‌سازی آن‌ها.
    \item 
    تعریف زیرمسئله‌ها: به منظور تسهیل توسعه، مسئله اصلی به چندین زیرمسئله تقسیم می‌شود که هر یک به طور جداگانه و به صورت ماژولار قابل حل باشد.
\end{itemize}


\subsubsection*{3. کدنویسی و توسعه محصول}

\begin{itemize}
    \item 
    پیاده‌سازی: پس از تایید طرح کلی، تیم توسعه کار بر روی ساخت و کدنویسی سیستم را آغاز می‌کند.
    \item 
    تطابق کد با طراحی اولیه: تیم توسعه، طراحی و معماری تعریف شده را به کد تبدیل می‌کند و از این طریق اطمینان حاصل می‌شود که کد با مدل‌های ارائه شده در مرحله قبلی مطابقت دارد.
    \item 
    بررسی و صحت‌سنجی هر بخش: با استفاده از بررسی‌های داخلی و استفاده از کدهای تست اولیه، تیم تلاش می‌کند تا هر مرحله از کدنویسی را مورد آزمایش قرار دهد. در صورت وجود خطا، تیم آن‌ها را اصلاح و مجدداً بررسی می‌کند.
    \item 
    آزمایش‌ها و تصحیح‌ها: از روش‌هایی مانند تست واحد برای ارزیابی اجزای کوچک و تست‌های یکپارچگی برای ارزیابی ارتباط بین ماژول‌ها استفاده می‌شود تا اطمینان حاصل شود که سیستم به‌درستی کار می‌کند و همه عملکردهای اساسی به‌طور دقیق اجرا می‌شوند.
\end{itemize}


\subsubsection*{4. تست و کنترل کیفیت}

\begin{itemize}
    \item 
    آزمایش و اطمینان از کیفیت: بررسی نهایی با هدف اطمینان از دقت و کارایی نرم‌افزار انجام می‌شود و معیارهای کیفیت به دقت کنترل می‌شود.
    \item 
    تست عملکرد سیستم: انجام تست‌های نهایی برای اطمینان از اینکه سیستم به تمام نیازهای تعریف شده در اسناد اولیه پاسخ می‌دهد.
    \item 
    ارزیابی کیفیت فنی و امنیتی: بررسی مواردی چون امنیت، کارایی و استحکام سیستم برای اطمینان از اینکه محصول نهایی برای استفاده در محیط واقعی آماده است.
    \item 
    برگزاری جلسه بازنگری با ذینفعان: نمایش محصول اولیه و دریافت بازخورد از طرف ذینفعان برای رفع مشکلات و برطرف کردن نیازهای احتمالی.
\end{itemize}

\subsubsection*{5. تحویل و استقرار محصول}

\begin{itemize}
    \item 
    انتقال و استقرار محصول: پس از انجام تمام تست‌ها و اطمینان از آمادگی کامل سیستم، محصول نهایی به شرکت تحویل داده می‌شود.
    \item 
    آموزش و پشتیبانی کاربران: آموزش تیم فنی شرکت و ارائه اسناد و دستورالعمل‌های لازم برای استفاده از سیستم.
    \item 
    استقرار در محیط عملیاتی: نصب و راه‌اندازی سیستم در محیط واقعی، اطمینان از عملکرد صحیح آن و حل مشکلات احتمالی که ممکن است در این مرحله بروز کنند.
    \item 
    پشتیبانی پس از استقرار: برای پشتیبانی از کاربران و رفع مشکلات احتمالی، تیم پشتیبانی باید به مدت معین پس از تحویل محصول در دسترس باشد و پشتیبانی فنی ارائه دهد.
\end{itemize}


\subsectionaddtolist{بخش دوم}

برای رسیدن به یک برنامه مناسب، ابتدا باید به این پرسش‌ها پاسخ دهیم:

\begin{itemize}
    \item چگونه می‌توان از عقب‌ماندگی کاست و رقابت را از دست نداد؟
    \item آیا می‌توان از قابلیت‌های فعلی پروژه بهره بیشتری برد؟
\end{itemize}
مسیر فکری و اقدامات پیشنهادی به شرح زیر هستند، این رویکرد نه تنها باعث تسریع در توسعه و عرضه محصول می‌شود، بلکه از دیدگاه مشتریان، محصولی نوآورانه و به‌روز به بازار عرضه می‌کند که توانایی رقابت با محصول رقیب را دارد:

\subsubsection*{1. ارزیابی وضعیت کنونی پروژه}

\textbf{مسیر فکری:}

\begin{itemize}
    \item ابتدا باید بدانیم پروژه تا چه اندازه به اهداف تعریف‌شده نزدیک است و چه میزان از آن تکمیل شده است. این امر به شناسایی نقاط ضعف و قوت ما در مقایسه با محصول رقیب کمک می‌کند.
\end{itemize}

\textbf{اقدامات:}

\begin{itemize}
    \item بررسی جزئیات پروژه در شش ماه گذشته: تهیه گزارشی از مراحل انجام شده، نقاط قوت و ضعف، موانع و دستاوردها.
    \item تحلیل قابلیت‌های محصول در مقایسه با محصول دیگر: در این مرحله باید ببینیم آیا ویژگی‌هایی که تاکنون طراحی کرده‌ایم، رقابتی هستند یا خیر. این تحلیل به ما کمک می‌کند که تصمیم بگیریم آیا به توسعه فعلی ادامه دهیم یا تغییراتی در ویژگی‌های کلیدی محصول اعمال کنیم.
\end{itemize}

\subsubsection*{2. بهبود و بازنگری در قابلیت‌ها}

\textbf{مسیر فکری:}

\begin{itemize}
    \item با توجه به محوریت هوش مصنوعی در محصول AIrpet ، یکی از اولویت‌های مهم این است که قابلیت‌های هوش مصنوعی پروژه را تقویت کنیم و امکانات نوآورانه‌تری نسبت به رقیب ارائه دهیم.
\end{itemize}

\textbf{اقدامات:}

\begin{itemize}
    \item ایجاد تیم تخصصی هوش مصنوعی: تشکیل تیمی کوچک و متخصص در حوزه هوش مصنوعی برای تمرکز بر ویژگی‌های اختصاصی و پیشرفته‌تر.
    \item همکاری با متخصصان و مشاوران AI خارجی یا داخلی: استفاده از مشاوران خارجی یا متخصصانی که پیشتر در پروژه‌های مشابه کار کرده‌اند، می‌تواند به شتاب‌دهی پروژه کمک کند و نوآوری‌های جدیدی را به محصول اضافه کند.
    \item افزودن قابلیت‌های هوش مصنوعی پیشرفته‌تر و شخصی‌سازی شده: بررسی و پیاده‌سازی قابلیت‌هایی نظیر تشخیص الگوهای فرش توسط هوش مصنوعی، تنظیمات تطبیق‌پذیر بر اساس سلیقه مشتری، و قابلیت پیشنهاد بر اساس بازخورد کاربر.
\end{itemize}


\subsubsection*{3. بازنگری و تسریع در فرآیند توسعه}

\textbf{مسیر فکری:}

\begin{itemize}
    \item با توجه به محدودیت زمانی و عقب‌ماندگی از رقیب، باید فرآیند توسعه را تسریع کرد. در این مسیر می‌توان از برخی روش‌های چابک و نوآورانه استفاده کرد.
\end{itemize}

\textbf{اقدامات:}

\begin{itemize}
    \item استفاده از روش‌های توسعه چابک(Agile) : برای افزایش سرعت توسعه، پیشنهاد می‌شود از روش‌های چابک مانند اسکرام (Scrum) بهره ببریم. تقسیم پروژه به «اسپرینت‌ها» و اهداف کوتاه‌مدت به‌جای تمرکز بر روی اهداف بلندمدت کمک می‌کند تا سریع‌تر به دستاوردهای قابل تحویل دست یابیم.
    \item تقسیم تیم‌ها و توزیع وظایف به صورت هم‌زمان: تجزیه و تقسیم پروژه به تیم‌های کوچک‌تر و تخصصی‌تر که بتوانند به صورت موازی کار کنند و برخی از وظایف را به طور هم‌زمان انجام دهند.
\end{itemize}


\subsubsection*{4. استفاده از قابلیت MVP}

\textbf{مسیر فکری:}
\begin{itemize}
    \item با توجه به محدودیت زمانی، می‌توان ابتدا یک محصول اولیه با قابلیت‌های اصلی به بازار عرضه کرد و پس از دریافت بازخورد، به توسعه و بهبود آن پرداخت.
\end{itemize}

\textbf{اقدامات:}

\begin{itemize}
    \item تعیین MVP با ویژگی‌های کلیدی: شناسایی مهم‌ترین و پرکاربردترین قابلیت‌های محصول و تمرکز روی توسعه و عرضه این قابلیت‌ها به صورت اولیه.
    \item آزمایش اولیه MVP و دریافت بازخورد کاربران: از طریق آزمایش و بازخورد سریع از کاربران، می‌توان محصول را بهبود بخشید و تغییرات لازم را سریع‌تر اعمال کرد.
\end{itemize}


\subsubsection*{5. کمپین بازاریابی و اطلاع‌رسانی رقابتی}

\textbf{مسیر فکری:}

\begin{itemize}
    \item با توجه به رونمایی محصول AIrpet ، باید اطلاع‌رسانی مناسبی از برنامه‌ها و ویژگی‌های منحصر به فرد محصول «مهر فرش گستران سنت» صورت گیرد تا مشتریان به این باور برسند که محصول جدید همچنان می‌تواند پاسخگوی نیازهای آنان باشد.
\end{itemize}

\textbf{اقدامات:}

\begin{itemize}
    \item راه‌اندازی کمپین اطلاع‌رسانی در مورد ویژگی‌های متمایز: اطلاع‌رسانی دقیق و تبلیغاتی که به وضوح ویژگی‌های متمایز محصول و مزایای رقابتی آن را در مقابل محصول رقیب به نمایش بگذارد.
    \item استفاده از بازخوردهای اولیه کاربران برای تبلیغ محصول: با جمع‌آوری و نمایش بازخوردهای مثبت کاربران اولیه محصول، می‌توان اعتماد بیشتری از مشتریان جدید کسب کرد.
    \item ایجاد یک طرح پیش‌فروش برای جلب توجه بازار: این اقدام می‌تواند باعث افزایش هیجان و استقبال بازار شود و مشتریان را به انتظار برای محصول نهایی ترغیب کند.
\end{itemize}


\subsubsection*{6. برنامه‌ریزی برای انتشار مداوم بهبودها و قابلیت‌های جدید}

\textbf{مسیر فکری:}

\begin{itemize}
    \item پس از عرضه اولیه محصول، باید به طور مداوم قابلیت‌ها و بهبودهای جدیدی به محصول اضافه شود تا جذابیت و کاربردپذیری محصول در طول زمان افزایش یابد.
\end{itemize}

\textbf{اقدامات:}

\begin{itemize}
    \item زمان‌بندی به‌روزرسانی‌های دوره‌ای: برای حفظ رقابت‌پذیری، برنامه‌ریزی منظمی برای انتشار به‌روزرسانی‌ها و قابلیت‌های جدید در بازه‌های زمانی مشخص صورت گیرد.
    \item جذب بازخورد کاربران و استفاده از آن در برنامه‌های توسعه بعدی: تحلیل نظرات و بازخوردهای کاربران در جهت افزودن قابلیت‌های کاربردی‌تر و حذف یا بهبود موارد ناکارآمد.
\end{itemize}
