\subsectionaddtolist{بخش اول}
\subsubsection*{شباهت‌ها}
\begin{enumerate}
\item توجه مستمر به ارزیابی و مدیریت ریسک در هر تکرار: 

هر دو مدل مارپیچی و فرایند یکپارچه به صورت تکرارشونده طراحی شده‌اند و در هر چرخه، ارزیابی ریسک صورت می‌گیرد. در مدل مارپیچی، هر دور از مارپیچ با ارزیابی ریسک همراه است، که این امر به اتخاذ تصمیم‌های مناسب قبل از حرکت به مرحله بعد کمک می‌کند​. در فرایند یکپارچه، هر تکرار شامل ارزیابی مشکلات و ریسک‌ها است که امکان کنترل و بهبود مستمر را فراهم می‌کند​.

\item ساختار انعطاف‌پذیر و واکنش‌پذیر نسبت به تغییرات و رفع مشکلات:

هر دو مدل با طراحی تکرارشونده امکان تطبیق با تغییرات و حل مشکلات پیش‌بینی‌نشده را فراهم می‌کنند. در پروژه‌هایی با نیازهای پویا و حساسیت بالا مانند سیستم پزشکی، این ویژگی باعث می‌شود که تیم بتواند با اطمینان بیشتری ریسک‌ها را مدیریت کند و در هر تکرار بازخورد لازم برای کاهش ریسک‌ها را به دست آورد.
\end{enumerate}

\subsubsection*{تفاوت‌ها}
\begin{enumerate}
    \item نحوه ارزیابی ریسک در طول فرآیند توسعه:
    \begin{itemize}
        \item مدل مارپیچی: این مدل بر ارزیابی مداوم و پیوسته ریسک‌ها در هر دور از چرخه توسعه تمرکز دارد. در هر تکرار، ارزیابی ریسک‌ها به عنوان بخش اصلی فرآیند مارپیچ انجام می‌شود و هرگونه تغییر در اولویت‌ها و تصمیم‌گیری‌ها مبتنی بر شناسایی ریسک‌های جدید است. این روش به تیم اجازه می‌دهد که در هر مرحله ریسک‌های به‌روز و مرتبط با هر فاز را شناسایی کرده و استراتژی مناسبی برای کاهش آنها انتخاب کند.
        \item فرایند یکپارچه: در این مدل، ارزیابی ریسک در مراحل اولیه پروژه مانند فاز "شروع" و "تکامل" انجام می‌شود، جایی که ریسک‌های اصلی و تاثیرگذار شناسایی شده و معماری و برنامه‌ریزی کلی با در نظر گرفتن این ریسک‌ها پی‌ریزی می‌شود. برخلاف مدل مارپیچی، در فرایند یکپارچه پس از مراحل اولیه، مدیریت ریسک به‌طور مستقیم در هر تکرار دنبال نمی‌شود و بیشتر به عنوان بخشی از فعالیت‌های پشتیبان در چارچوب کلی پروژه در نظر گرفته می‌شود.
    \end{itemize}
    \item نقش مدیریت ریسک در تصمیم‌گیری‌ها و جهت‌دهی به پروژه:
    \begin{itemize}
        \item مدل مارپیچی: مدیریت ریسک در این مدل به عنوان عنصر مرکزی و راهبردی در هر تکرار عمل می‌کند و تصمیمات کلیدی برای ادامه یا تغییر مسیر در پایان هر چرخه بر اساس ارزیابی ریسک‌های شناسایی‌شده اتخاذ می‌شود. این امر موجب می‌شود که پروژه به‌طور پویا به تغییرات پاسخ دهد و تیم بتواند در هر مرحله بر اساس تحلیل دقیق ریسک‌ها اقدامات اصلاحی را انجام دهد.
        \item فرایند یکپارچه: در این مدل، مدیریت ریسک در تصمیم‌گیری‌ها و برنامه‌ریزی‌های اولیه اهمیت دارد، اما پس از تعیین چارچوب معماری و ساختار پروژه در مراحل اولیه، نقش مستقیم آن کاهش می‌یابد. تصمیمات پروژه عمدتاً بر اساس طراحی و معماری تثبیت‌شده اتخاذ می‌شود و مدیریت ریسک به عنوان یک فعالیت پشتیبان در بهبود مستمر کیفیت و تغییرات جزئی محصول در طول چرخه زندگی پروژه باقی می‌ماند.
    \end{itemize}
\end{enumerate}
\subsectionaddtolist{بخش دوم}

\subsubsection*{مدل مارپیچی}
\begin{itemize}
    \item شناسایی ریسک: 
    
    هر دور از چرخه مارپیچی با ارزیابی و شناسایی ریسک‌ها آغاز می‌شود. این شناسایی شامل بررسی ریسک‌های فنی، زمانی، مالی، و عملکردی است که ممکن است بر روند توسعه تأثیر بگذارد. تیم توسعه به طور مستمر در هر تکرار به دنبال شناسایی ریسک‌های جدید و بررسی عوامل بالقوه‌ای است که ممکن است تهدیدی برای موفقیت پروژه باشند.
    \item تحلیل ریسک:
    
    پس از شناسایی ریسک‌ها، مدل مارپیچی بر تحلیل عمیق این ریسک‌ها تمرکز دارد. تیم توسعه در هر تکرار میزان تأثیر و احتمال وقوع هر ریسک را ارزیابی می‌کند. این تحلیل به تیم کمک می‌کند که ریسک‌ها را اولویت‌بندی کند و منابع بیشتری را به ریسک‌های مهم‌تر و محتمل‌تر اختصاص دهد.
    \item کاهش ریسک:
    
    در این مرحله، بر اساس نتایج تحلیل، تیم به دنبال طراحی راه‌حل‌هایی برای کاهش یا رفع ریسک‌های شناسایی‌شده است. اقدامات کاهش‌دهنده می‌تواند شامل آزمایشات اولیه، ساخت نمونه‌های اولیه و استفاده از تکنیک‌های خاص برای بررسی صحت عملکرد باشد. در هر تکرار، این مدل به تیم اجازه می‌دهد که هر مرحله را با تمرکز بر کاهش ریسک تکمیل کند، پیش از آنکه به مراحل بعدی برود. این فرآیند به خصوص برای پروژه‌هایی با ریسک‌های زیاد و متغیر، مانند سیستم‌های حساس به داده‌های پزشکی، مفید است.
\end{itemize}
\subsubsection*{مدل فرایند یکپارچه}
\begin{itemize}
    \item شناسایی ریسک:
    
    در فازهای اولیه مانند فاز شروع و تکامل، فرایند یکپارچه به شناسایی ریسک‌های کلی و اساسی پروژه می‌پردازد. این شناسایی شامل ریسک‌های مرتبط با نیازهای مشتری، محدودیت‌های فنی و مشکلات احتمالی در پیاده‌سازی می‌شود. در این مراحل، تمرکز اصلی بر شناسایی ریسک‌هایی است که می‌توانند در بلندمدت بر کل پروژه تاثیر بگذارند.
    \item تحلیل ریسک:
    
    فرایند یکپارچه به تحلیل ریسک‌ها در فازهای آغازین پرداخته و با استفاده از این تحلیل، چارچوب و معماری پروژه را طراحی می‌کند. این تحلیل به تیم توسعه کمک می‌کند تا نیازها و محدودیت‌های اصلی پروژه را به درستی ارزیابی کرده و معماری سیستم را به گونه‌ای طراحی کند که ریسک‌های کلیدی را از ابتدا کنترل کند.
    \item کاهش ریسک:
    
    پس از تحلیل، فرایند یکپارچه از طریق ایجاد یک معماری قابل اعتماد و پایدار در مراحل اولیه به کاهش ریسک‌ها می‌پردازد. علاوه بر این، در هر تکرار به بررسی مجدد و بهینه‌سازی معماری و طراحی سیستم پرداخته می‌شود. اما برخلاف مدل مارپیچی، فرایند یکپارچه به طور مداوم بر شناسایی و کاهش ریسک‌ها در طول هر تکرار تمرکز ندارد، بلکه تمرکز اصلی آن بر طراحی صحیح و جامع اولیه است که بتواند بیشتر ریسک‌ها را در درازمدت کاهش دهد.
\end{itemize}
\subsectionaddtolist{بخش سوم}

با توجه به نیاز به تطابق با استانداردهای سخت‌گیرانه، که محافظت از حریم خصوصی و امنیت داده‌های بیماران را الزامی می‌کند، مدل مارپیچی گزینه مناسب‌تری برای این پروژه خواهد بود. زیرا:

\subsubsection*{مدیریت دقیق و پیوسته ریسک}
مدل مارپیچی بر اساس یک چرخه تکرارشونده طراحی شده است که در هر دور آن، ارزیابی و کاهش ریسک‌های جدید انجام می‌شود. برای پروژه‌ای که شامل داده‌های حساس پزشکی و نیازمند حفظ حریم خصوصی بیماران است، این ویژگی بسیار مهم است. این مدل به تیم توسعه امکان می‌دهد تا در هر تکرار:

\begin{itemize}
    \item ریسک‌های جدید را شناسایی و تحلیل کند.
    \item ریسک‌های مربوط به امنیت و حفظ حریم خصوصی داده‌های بیماران را اولویت‌بندی و اقدامات لازم را برای کاهش آنها اتخاذ کند.
\end{itemize}

این بررسی‌های دوره‌ای، انعطاف لازم را برای انطباق با تغییرات احتمالی در استانداردها و قوانین امنیتی و حفظ حریم خصوصی فراهم می‌کند و باعث می‌شود تیم بتواند به سرعت به هر گونه تهدید یا ریسک جدید در طول چرخه توسعه واکنش نشان دهد.



\subsubsection*{انعطاف‌پذیری در پیاده‌سازی امنیت و حریم خصوصی}
مدل مارپیچی به دلیل تمرکز بر کاهش ریسک در هر مرحله، به تیم اجازه می‌دهد که برای موارد خاص مانند تطابق با HIPAA یا کنترل‌های امنیتی خاص، اقدامات خاصی را در هر تکرار به کار گیرد. این انعطاف‌پذیری به تیم کمک می‌کند تا در هر دور از چرخه، ویژگی‌های امنیتی و حفظ حریم خصوصی را بررسی کرده و در صورت نیاز آنها را بهبود بخشد. از آنجا که این پروژه نیاز به حفاظت از داده‌های حساس بیماران دارد، توجه مستمر به حریم خصوصی و امنیت در هر مرحله از توسعه می‌تواند از بروز آسیب‌پذیری‌های احتمالی جلوگیری کند.
\subsectionaddtolist{بخش چهارم}

در مدیریت تغییرات و استقرارهای مکرر، هر دو مدل افزایشی و چابک قابلیت‌ها و رویکردهای خاص خود را دارند:

\subsubsection*{مدل افزایشی}
\begin{itemize}
    \item مدیریت تغییرات: در مدل افزایشی، سیستم به‌صورت تدریجی و در چندین مرحله ساخته و توسعه داده می‌شود. هر بخش از سیستم به‌عنوان یک افزایش تعریف می‌شود که شامل ویژگی‌ها یا بخش‌هایی از عملکرد کلی پروژه است. در صورت بروز تغییرات جدید، می‌توان این تغییرات را در یکی از این "افزایش‌ها" برنامه‌ریزی کرد و سپس آن را در سیستم اصلی ادغام نمود. این مدل امکان مدیریت تغییرات را فراهم می‌کند، اما تغییرات جدید معمولاً پس از پایان هر افزایش و قبل از شروع افزایش بعدی در نظر گرفته می‌شود. بنابراین، مدیریت تغییرات در این مدل از انعطاف کمتری نسبت به مدل چابک برخوردار است.
    \item مدیریت استقرارهای مکرر: هر افزایش در مدل افزایشی به‌صورت مستقل توسعه و آزمایش می‌شود و پس از تکمیل، به سیستم اصلی افزوده می‌شود. استقرارهای مکرر در این مدل به این صورت انجام می‌شود که هر افزایش تکمیل شده به محصول نهایی اضافه و استقرار داده می‌شود. این روش برای پروژه‌هایی که نیاز به استقرار‌های گام به گام دارند مناسب است، اما سرعت استقرار‌های آن معمولاً کندتر از مدل چابک است زیرا باید هر افزایش به طور کامل تکمیل شود.
\end{itemize}
مدل افزایشی برای پروژه‌هایی که نیاز به استقرارهای مرحله‌ای و تغییرات کمتر دارند مناسب است. در این مدل تغییرات و به‌روزرسانی‌ها پس از تکمیل هر افزایش اعمال می‌شوند و استقرارهای مکرر به اندازه مدل چابک انعطاف‌پذیر نیست.
\subsubsection*{مدل چابک}
\begin{itemize}
    \item مدیریت تغییرات: در مدل چابک، تغییرات به عنوان بخشی از فرآیند توسعه پذیرفته و حتی تشویق می‌شوند. تیم‌ها از بازه‌های کوتاهی به نام اسپرینت استفاده می‌کنند که در هر اسپرینت، می‌توان تغییرات جدید را بررسی و در صورت نیاز، در برنامه‌ها اعمال کرد. این انعطاف‌پذیری بالا باعث می‌شود که تغییرات به‌طور پیوسته و بدون نیاز به بازنگری کل سیستم در نظر گرفته شوند و ویژگی‌های جدید به سرعت به محصول اضافه شوند.
    \item مدیریت استقرارهای مکرر: مدل چابک به دلیل استفاده از اسپرینت‌های کوتاه و توسعه مداوم، امکان استقرارهای مکرر و سریع را فراهم می‌کند. در پایان هر اسپرینت، یک نسخه عملیاتی و قابل استفاده از سیستم که شامل تغییرات و به‌روزرسانی‌های جدید است، ارائه می‌شود. این رویکرد استقرار مداوم به خصوص برای پروژه‌هایی با نیازهای متغیر و پویا مانند سیستم‌های پزشکی مناسب است که در آن‌ها به‌روزرسانی‌های سریع و رفع مشکلات بدون اختلال در عملکرد اصلی سیستم ضروری است.
\end{itemize}
مدل چابک به دلیل انعطاف بالا در مدیریت تغییرات و سرعت استقرارهای مکرر، برای پروژه‌های پویا که نیاز به تطبیق سریع با نیازهای جدید و به‌روزرسانی‌های امنیتی دارند، ایده‌آل است.
\subsectionaddtolist{بخش پنجم}
این بخش از مسئله در بخش واضح (Clear) چارچوب کانوین جای می‌گیرد. بخش واضح در چارچوب کانوین برای پروژه‌هایی مناسب است که راه‌حل‌های از پیش‌تعیین‌شده و قابل اطمینان دارند، نیاز به تحلیل پیچیده یا اکتشاف ندارند و از رویکردهای استاندارد و شناخته‌شده استفاده می‌کنند.
\begin{itemize}
    \item شفافیت نیازمندی‌ها:
    
    در این پروژه، نیازمندی‌ها از پیش به‌طور کامل مشخص و شفاف هستند و سال‌هاست که ثبات دارند. این بدان معناست که هیچ ابهام یا تغییر غیرمنتظره‌ای در خواسته‌های سیستم وجود ندارد و تیم به طور دقیق می‌داند چه چیزی باید پیاده‌سازی شود.
    \item تجربه قبلی تیم:
    
    تیم فنی پروژه تجربه بالایی در زمینه پیاده‌سازی سیستم‌های مشابه دارد و پیش‌تر این نوع پروژه‌ها را انجام داده است. بنابراین، راه‌حل‌ها و فرآیندهای لازم برای اجرای پروژه به طور کامل آشنا هستند.

    \item دسترسی به روالهای مطلوب:
    
    در این پروژه، منابع کافی از روش‌های بهینه و روال‌های مطلوب برای پیاده‌سازی سیستم وجود دارد. این ویژگی مشخصاً نشان می‌دهد که بهترین راه‌حل‌ها از قبل در دسترس هستند و نیازی به کشف یا ابداع فرآیندهای جدید نیست.
\end{itemize}
\subsectionaddtolist{بخش ششم}


برای این پروژه خاص، مدل آبشاری به دلایل زیر بهترین گزینه است:
\begin{itemize}
    \item نیازمندی‌های شفاف و پایدار: 
    
    در مدل آبشاری، مراحل به‌صورت متوالی و خطی پیش می‌روند و هر مرحله پس از تکمیل مرحله قبلی آغاز می‌شود. این مدل برای پروژه‌هایی مناسب است که نیازمندی‌ها از ابتدا مشخص و پایدار هستند و به تغییرات زیادی نیاز ندارند. از آنجا که نیازمندی‌های این پروژه از سال‌ها پیش مشخص و شفاف بوده و احتمال تغییر در آنها بسیار کم است، مدل آبشاری به دلیل تمرکز بر طراحی و برنامه‌ریزی دقیق در مراحل اولیه، می‌تواند گزینه‌ای مؤثر باشد.
    \item تجربه قبلی تیم و وجود روال‌های مطلوب:
    
    با توجه به اینکه تیم تجربه بالایی در اجرای پروژه‌های مشابه دارد و منابع کافی از روال‌های مطلوب برای این سیستم در دسترس است، می‌توان از مدل آبشاری که رویکردی ساختارمند و استاندارد دارد استفاده کرد. این مدل با پیروی از روش‌های تعریف‌شده در مراحل مختلف، می‌تواند به تیم کمک کند تا با بهره‌گیری از دانش و تجربه قبلی به‌طور مؤثر به اهداف پروژه دست یابد.
    \item عدم نیاز به تکرار یا انعطاف بالا:
    
    مدل آبشاری به دلیل ساختار خطی خود، انعطاف‌پذیری بالایی برای تغییرات مداوم ندارد. اما در این پروژه به دلیل پایداری نیازمندی‌ها و شفافیت آن‌ها، نیازی به تغییرات مکرر یا بازبینی‌های پیوسته وجود ندارد. از این رو، مدل آبشاری می‌تواند فرآیندی ساده و منظم برای تکمیل پروژه فراهم کند.
\end{itemize}

