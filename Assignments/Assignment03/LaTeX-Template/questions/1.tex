\subsectionaddtolist{الف}




	در این سناریو، الگویی که می‌تواند کمک کند الگوی مدارشکن (Circuit-Breaker-Pattern) است. این الگو به گونه‌ای طراحی شده است که با شناسایی خطاهای مکرر یا تاخیرهای طولانی در پاسخ‌دهی، ارتباط با سرویس معیوب را به طور موقت قطع کند و از ارسال درخواست‌های جدید به سرویس معیوب جلوگیری نماید.
	
	\textbf{مزایای الگوی مدارشکن:}
	\begin{itemize}
		\item جلوگیری از مشکلات آبشاری: با قطع موقت ارتباط، بار اضافی از روی سرویس معیوب برداشته شده و از گسترش مشکلات به سایر سرویس‌ها جلوگیری می‌شود.
		\item افزایش تاب‌آوری سیستم: به دلیل کاهش وابستگی مستقیم به سرویس معیوب، سرویس‌های دیگر می‌توانند به عملکرد خود ادامه دهند.
		\item بازیابی تدریجی: مدارشکن به صورت دوره‌ای وضعیت سرویس معیوب را بررسی می‌کند و در صورت بهبود، ارتباط مجدد را امکان‌پذیر می‌سازد.
		\item مدیریت بهتر منابع: با جلوگیری از ارسال درخواست‌های بی‌فایده به سرویس معیوب، منابع سیستم بهینه‌تر استفاده می‌شود.
	\end{itemize}
	
	الگوی مدارشکن می‌تواند به عنوان یک بخش مهم از طراحی معماری میکروسرویس به کار گرفته شود تا تاب‌آوری و پایداری سیستم را افزایش دهد.


\subsectionaddtolist{ب}


	\begin{enumerate}
		\item \textbf{استفاده از داده‌های کش:} اطلاعات موجودی کالاها را در یک سیستم کش قابل اطمینان ذخیره کنید. در صورت عدم دسترسی به سرویس «مدیریت موجودی»، از این داده‌ها برای ارائه پاسخ به کاربران استفاده شود.
		\item \textbf{پاسخ‌های Fallback :} هنگام خرابی سرویس، پاسخ‌های پیش‌فرضی برای کاربران ارائه شود؛ مثلاً نمایش پیامی مانند «اطلاعات موجودی در حال حاضر در دسترس نیست، لطفاً بعداً تلاش کنید».
		\item \textbf{به‌روزرسانی غیر هم‌زمان موجودی:} در صورتی که امکان بررسی موجودی در لحظه وجود ندارد، سفارش را ثبت کرده و عملیات بررسی موجودی و تأیید نهایی پرداخت را به صورت غیرهم‌زمان مدیریت کنید.
		\item \textbf{نمایش پیغام شفاف به کاربران:} به کاربران اعلام کنید که سیستم با اختلال مواجه است و آن‌ها را از وضعیت به‌روز نگه دارید. این شفافیت به کاهش نارضایتی کاربران کمک می‌کند.
		\item \textbf{مانیتورینگ و اطلاع‌رسانی:} با استفاده از ابزارهای مانیتورینگ، وضعیت سرویس‌ها را نظارت کنید و در صورت خرابی یا تأخیر، به تیم‌های مربوطه اطلاع دهید تا مشکلات سریع‌تر رفع شوند.
		\item \textbf{طراحی تجربه‌ی کاری جایگزین:} در صورت خرابی سرویس، به کاربران امکان مشاهده‌ی سایر اطلاعات یا محصولات مشابه را بدهید تا آن‌ها همچنان بتوانند تجربه‌ی مثبتی از سیستم داشته باشند.
		\item \textbf{سیستم انتظار و رزرو:} در صورت عدم دسترسی به اطلاعات موجودی، امکان رزرو کالا برای کاربر فراهم شود و پس از رفع مشکل، وضعیت سفارش به‌روز شود.
	\end{enumerate}
	
	این اقدامات می‌تواند تجربه‌ی کاربران را حتی در صورت وقوع خرابی حفظ کند و از کاهش رضایت یا اعتماد آن‌ها به سیستم جلوگیری نماید.

\subsectionaddtolist{ج}



	برای مدیریت و نظارت بهتر بر وضعیت سرویس معیوب و بازگرداندن ارتباط به حالت عادی، می‌توان از راه‌حل‌های زیر استفاده کرد:
	
	\textbf{۱. مانیتورینگ سلامت سرویس:} 
	یک سیستم مانیتورینگ پیشرفته برای نظارت مداوم بر وضعیت سرویس معیوب پیاده‌سازی کنید. ابزارهایی مانند \lr{Prometheus} و \lr{Grafana} می‌توانند شاخص‌های کلیدی عملکرد (\lr{KPIs}) مانند نرخ خطا، زمان پاسخ‌دهی، و نرخ موفقیت درخواست‌ها را پایش کنند. با این روش، هرگونه اختلال سریعاً شناسایی شده و تیم‌های فنی می‌توانند اقدامات لازم را انجام دهند.
	
	\textbf{۲. استفاده از مکانیزم بازگرداندن تدریجی (Circuit-Breaker-Recovery) :} 
	مدارشکن می‌تواند به صورت دوره‌ای وضعیت سرویس معیوب را بررسی کند. به این صورت که پس از یک دوره‌ی انتظار، مدار به حالت نیمه‌باز تغییر می‌کند و تعداد محدودی درخواست آزمایشی به سرویس ارسال می‌شود. اگر پاسخ‌ها موفقیت‌آمیز بودند، مدارشکن به حالت بازگشته و ارتباط سرویس به صورت کامل بازیابی می‌شود. این مکانیزم امکان بازگشت تدریجی و پایدار سرویس را فراهم می‌آورد.
	
	این دو راه‌حل در کنار هم می‌توانند به تشخیص سریع مشکلات، کاهش زمان خرابی، و بازگرداندن ارتباط به حالت عادی کمک کنند.

