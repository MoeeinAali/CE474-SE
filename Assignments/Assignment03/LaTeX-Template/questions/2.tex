\subsectionaddtolist{الف}

\subsection*{کیفیت عملکرد (Performance-Quality)}
\paragraph{اقدامات ارزیابی:}
\begin{itemize}
	\item \textbf{تست بار (Load-Testing) :} شبیه‌سازی شرایطی که هزاران کاربر همزمان از نرم‌افزار استفاده می‌کنند تا توانایی پاسخگویی سیستم سنجیده شود.
	\item \textbf{تست استرس (Stress-Testing) :} بررسی عملکرد نرم‌افزار در شرایطی که بار از حد معمول بالاتر است (تا مرز شکست سیستم) برای شناسایی نقاط ضعف.
	\item \textbf{تست مقیاس‌پذیری (Scalability-Testing) :} ارزیابی توانایی نرم‌افزار برای افزایش یا کاهش منابع برای مدیریت تعداد کاربران بیشتر.
	\item \textbf{مانیتورینگ زمان پاسخ (Response-Time-Monitoring) :} اندازه‌گیری زمان پاسخ‌دهی به درخواست‌های کاربران و شناسایی مواردی که از حد مجاز کندتر هستند.
\end{itemize}

\paragraph{ابزارها:}
استفاده از ابزارهایی مانند \textbf{JMeter}، \textbf{LoadRunner} یا \textbf{Apache-Benchmark} برای انجام آزمون‌ها.

\subsection*{کیفیت تطبیق (Compliance-Quality)}
\paragraph{اقدامات ارزیابی:}
\begin{itemize}
	\item \textbf{تست امنیت (Security-Testing) :} بررسی آسیب‌پذیری‌های امنیتی مانند تزریق SQL ، حملات XSS ، و ضعف‌های احراز هویت.
	\item \textbf{تست تطابق قانونی (Regulatory-Compliance-Testing) :} ارزیابی انطباق با استانداردها و قوانین بانکی مانند GDPR، PCI DSS، و ISO 27001.
	\item \textbf{بررسی احراز هویت و دسترسی :} اطمینان از اینکه مکانیسم‌های احراز هویت قوی و سطوح دسترسی مناسب برای کاربران مختلف وجود دارد.
	\item \textbf{بررسی ثبت رویدادها (Audit-Logging) :} اطمینان از اینکه همه رویدادهای حساس مانند ورود به سیستم یا تراکنش‌ها ثبت و نگهداری می‌شوند.
\end{itemize}

\paragraph{ابزارها:}
استفاده از ابزارهای تست امنیت مانند \textbf{OWASP-ZAP}، \textbf{Burp-Suite}، یا \textbf{Nessus}. همچنین بهره‌گیری از چک‌لیست‌های قانونی برای تطبیق با قوانین و استانداردها.

\subsection*{تفاوت بین کیفیت عملکرد و کیفیت تطبیق}
\begin{itemize}
	\item \textbf{کیفیت عملکرد:} بر توانایی نرم‌افزار برای ارائه خدمات بهینه در شرایط مختلف (مانند بار زیاد) تمرکز دارد.
	\item \textbf{کیفیت تطبیق:} به رعایت استانداردهای امنیتی و قانونی مربوط به داده‌های بانکی و کاربران ارتباط دارد.
\end{itemize}


\subsectionaddtolist{ب}


\subsubsection*{1. آزمون‌های کیفیت عملکرد (Performance-Quality-Tests)}
\begin{itemize}
	\item \textbf{تست بار (Load-Testing) :} بررسی عملکرد سیستم تحت بار عادی و پیش‌بینی‌شده برای اطمینان از پاسخ‌دهی مناسب.
	\item \textbf{تست استرس (Stress-Testing) :} اعمال بار بیشتر از حد معمول برای شناسایی نقاط ضعف سیستم و اندازه‌گیری میزان تحمل آن.
	\item \textbf{تست مقیاس‌پذیری (Scalability-Testing) :} ارزیابی توانایی سیستم در افزایش منابع (مانند سرورها) برای مدیریت کاربران بیشتر.
	\item \textbf{تست زمان پاسخ (Response-Time-Testing):} بررسی مدت زمانی که سیستم برای پاسخ به درخواست‌ها صرف می‌کند.
	\item \textbf{تست ظرفیت (Capacity-Testing) :} تعیین حداکثر تعداد کاربران یا حجم تراکنش‌هایی که سیستم می‌تواند پشتیبانی کند.
\end{itemize}

\subsubsection*{2. آزمون‌های کیفیت تطبیق (Compliance-Quality-Tests)}
\begin{itemize}
	\item \textbf{تست امنیت (Security-Testing) :} شناسایی آسیب‌پذیری‌ها مانند تزریق SQL، حملات XSS، و نقص‌های احراز هویت.
	\item \textbf{تست تطابق قانونی (Regulatory-Compliance-Testing) :} اطمینان از رعایت استانداردها و مقررات مانند \textbf{GDPR}، \textbf{PCI DSS}، و \textbf{27001ISO}.
	\item \textbf{تست نفوذ (Penetration-Testing) :} شبیه‌سازی حملات برای ارزیابی میزان مقاومت سیستم در برابر تهدیدهای امنیتی.
	\item \textbf{بررسی رمزنگاری (Encryption-Testing) :} ارزیابی الگوریتم‌ها و مکانیسم‌های رمزنگاری برای حفاظت از داده‌ها.
	\item \textbf{تست ثبت وقایع (Audit-Log-Testing) :} اطمینان از ثبت دقیق و کامل فعالیت‌های کاربران و تراکنش‌های حساس.
\end{itemize}

\subsubsection*{ابزارهای مرتبط:}
\begin{itemize}
	\item برای \textbf{کیفیت عملکرد:} ابزارهایی مانند \textbf{JMeter}، \textbf{LoadRunner}، و \textbf{Apache-Benchmark}.
	\item برای \textbf{کیفیت تطبیق:} ابزارهایی مانند \textbf{OWASP-ZAP}، \textbf{Burp Suite}، و \textbf{Nessus}.
\end{itemize}

\pagebreak

\subsectionaddtolist{ج}

\begin{itemize}
	\item \textbf{کیفیت عملکرد (Performance-Quality) :}
	کیفیت عملکرد به توانایی نرم‌افزار در ارائه خدمات بهینه و کارآمد تحت شرایط مختلف، از جمله بار زیاد، تمرکز دارد. این نوع کیفیت با عوامل زیر ارزیابی می‌شود:
	\begin{itemize}
		\item سرعت پاسخ‌دهی (Response-Time)
		\item مقیاس‌پذیری (Scalability)
		\item تحمل بار زیاد (Load-Tolerance)
		\item پایداری در شرایط بحرانی (System-Stability)
	\end{itemize}
	عدم توجه به این کیفیت می‌تواند منجر به کندی یا از کار افتادن سیستم شود.
	
	\item \textbf{کیفیت تطبیق (Compliance-Quality) :}
	کیفیت تطبیق به انطباق نرم‌افزار با استانداردهای امنیتی، قانونی، و صنعتی مرتبط است. این نوع کیفیت با عوامل زیر ارزیابی می‌شود:
	\begin{itemize}
		\item رعایت قوانین و مقررات
		\item امنیت داده‌ها و حفاظت از حریم خصوصی کاربران
		\item ثبت و گزارش‌گیری دقیق وقایع (Audit-Logging)
		\item مدیریت صحیح دسترسی‌ها (Access-Management)
	\end{itemize}
	عدم رعایت این کیفیت می‌تواند منجر به جرایم قانونی، نقض حریم خصوصی، و کاهش اعتماد کاربران شود.
\end{itemize}

\subsubsection*{خلاصه تفاوت:}
\begin{itemize}
	\item \textbf{کیفیت عملکرد} به بهره‌وری و کارایی سیستم در پاسخ به کاربران تمرکز دارد.
	\item \textbf{کیفیت تطبیق} به رعایت استانداردها و تضمین امنیت و قانونی بودن نرم‌افزار ارتباط دارد.
\end{itemize}

\subsectionaddtolist{د}


\subsubsection*{1. عدم رعایت کیفیت عملکرد (Performance-Quality) :}
\begin{itemize}
	\item \textbf{کندی سیستم:} زمان پاسخ‌دهی طولانی می‌تواند باعث ایجاد نارضایتی و از دست رفتن اعتماد کاربران شود.
	\item \textbf{قطع ارتباط یا خرابی:} اگر نرم‌افزار تحت بار زیاد از کار بیفتد، کاربران ممکن است قادر به تکمیل تراکنش‌های خود نباشند، که این موضوع باعث کاهش تجربه کاربری می‌شود.
	\item \textbf{عدم پایداری:} خرابی‌های مکرر یا ناپایداری سیستم ممکن است کاربران را به سمت استفاده از سرویس‌های رقیب سوق دهد.
\end{itemize}

\subsubsection*{2. عدم رعایت کیفیت تطبیق (Compliance-Quality) :}
\begin{itemize}
	\item \textbf{نقض حریم خصوصی:} افشای داده‌های حساس کاربران مانند اطلاعات بانکی یا شخصی می‌تواند منجر به کاهش اعتماد کاربران به نرم‌افزار شود.
	\item \textbf{نقض قوانین:} عدم رعایت مقررات مانند \textbf{GDPR} یا \textbf{PCI-DSS} ممکن است به جریمه‌های سنگین و بدنامی برند منجر شود.
	\item \textbf{ناامنی داده‌ها:} کاربران ممکن است احساس کنند که اطلاعات آن‌ها در معرض خطر قرار دارد و از استفاده از نرم‌افزار صرف‌نظر کنند.
\end{itemize}

\subsubsection*{نتیجه کلی:}

	 \textbf{کیفیت عملکرد:} مشکلات مرتبط با عملکرد می‌تواند منجر به نارضایتی آنی کاربران و کاهش استفاده از نرم‌افزار شود.
	
	\textbf{کیفیت تطبیق:} عدم رعایت تطبیق می‌تواند اثرات طولانی‌مدت‌تری مانند از دست رفتن اعتبار، مشکلات قانونی، و کاهش اعتماد عمومی ایجاد کند.


