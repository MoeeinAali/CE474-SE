\subsectionaddtolist{الف}

\begin{flushright}

	
	استفاده از اصول انتزاع تنها محدود به سطح کدهای پروژه نیست و می‌تواند در لایه‌های مختلف سیستم نرم‌افزاری، از جمله مدیریت داده‌ها، مورد استفاده قرار گیرد. انتزاع به معنای پنهان‌سازی جزئیات غیرضروری و تمرکز بر مفاهیم یا عملیات اصلی است. در سیستم مدیریت سفارش فودلاین، انتزاع می‌تواند در سطوح زیر اعمال شود:
	
	\textbf{۱. انتزاع در سطح کد:}
	\begin{itemize}
		\item طراحی ماژول‌ها و کلاس‌ها با استفاده از الگوهای طراحی (مانند \lr{Facade} یا \lr{Adapter}) برای پنهان‌سازی پیچیدگی و ارائه‌ی رابط‌های ساده.
		\item جداسازی منطق کسب‌وکار (Business-Logic) از لایه‌های دیگر، مانند رابط کاربری و دسترسی به داده‌ها، به منظور تسهیل در نگهداری و توسعه.
		\item استفاده از لایه‌های سرویس (Service-Layers) برای انتزاع منطق پیچیده و ارائه‌ی عملکردهای ساده و شفاف به لایه‌های بالاتر.
	\end{itemize}
	
	\textbf{۲. انتزاع در سطح داده‌ها:}
	\begin{itemize}
		\item \textbf{طراحی پایگاه داده انتزاعی:} به جای دسترسی مستقیم به جداول پایگاه داده، از مدل‌های داده‌ای و واسط‌هایی (مانند ORM ‌ها نظیر \lr{Entity-Framework} یا \lr{Hibernate}) استفاده کنید تا جزئیات نحوه ذخیره‌سازی داده‌ها پنهان شوند.
		\item \textbf{انتزاع در مدل‌سازی داده‌ها:} تعریف موجودیت‌های مفهومی مانند \lr{Order}، \lr{Customer} و \lr{MenuItem} به جای نمایش مستقیم جداول پایگاه داده.
		\item \textbf{پنهان‌سازی فرمت داده‌ها:} ارائه اطلاعات مشتریان یا سفارش‌ها به صورت انتزاعی و استاندارد، بدون نمایش جزئیات ساختار داده‌ها یا فرمت‌های داخلی مانند XML یا JSON .
	\end{itemize}
	
	\textbf{۳. انتزاع در سطح معماری:}
	\begin{itemize}
		\item جداسازی لایه‌های مختلف سیستم (مانند لایه‌های ارائه، منطق کسب‌وکار و داده) به گونه‌ای که تغییرات در یک لایه تأثیری بر لایه‌های دیگر نداشته باشد.
		\item استفاده از APIها برای ارائه خدمات به ماژول‌های داخلی یا سیستم‌های خارجی بدون نمایش جزئیات داخلی.
	\end{itemize}
	
	\textbf{نتیجه‌گیری:} 
	اصل انتزاع نه تنها در کدنویسی، بلکه در طراحی داده‌ها و معماری سیستم نیز قابل اجراست و می‌تواند به ساده‌سازی، پایداری و قابلیت نگهداری پروژه کمک کند.
\end{flushright}

\pagebreak

\subsectionaddtolist{ب}


\begin{flushright}

	\textbf{۱. طراحی ماژولار و میکروسرویس:}
	\begin{itemize}
		\item هر بخش از سیستم به صورت یک زیرسیستم مستقل طراحی شود که وظیفه مشخص و محدودی دارد (مانند زیرسیستم مدیریت سفارش‌ها، ردیابی وضعیت سفارش، و مدیریت اطلاعات مشتری).
		\item زیرسیستم‌ها به صورت مستقل از یکدیگر توسعه، اجرا و استقرار داده شوند.
		\item ارتباط میان زیرسیستم‌ها از طریق APIهای استاندارد یا پیام‌محور (\lr{Message Queues}) انجام شود تا وابستگی‌ها کاهش یابد.
	\end{itemize}
	
	\textbf{۲. استفاده از اصول SOLID در طراحی:}
	\begin{itemize}
		\item اصل \lr{Single-Responsibility}: هر زیرسیستم فقط یک مسئولیت مشخص داشته باشد (مثلاً زیرسیستم مدیریت سفارش تنها مسئول ذخیره و پردازش سفارش‌ها باشد).
		\item اصل \lr{Interface-Segregation}: زیرسیستم‌ها فقط وابسته به رابط‌های ضروری باشند و نیازی به استفاده از جزئیات غیرضروری دیگر سیستم‌ها نداشته باشند.
	\end{itemize}
	
	\textbf{۳. پیاده‌سازی ارتباط‌های غیرهم‌زمان:}
	\begin{itemize}
		\item استفاده از مکانیزم‌های غیرهم‌زمان مانند \lr{Kafka} یا \lr{RabbitMQ} برای ارسال و دریافت پیام‌ها بین زیرسیستم‌ها.
		\item این روش امکان جداسازی زمانی زیرسیستم‌ها و افزایش مقیاس‌پذیری را فراهم می‌کند.
	\end{itemize}
	
	\textbf{۴. طراحی برای شکست:}
	\begin{itemize}
		\item استفاده از الگوهایی مانند مدارشکن (\lr{Circuit-Breaker}) برای شناسایی و جلوگیری از خطاهای آبشاری در سیستم.
		\item پشتیبان‌گیری و بازیابی اطلاعات در زیرسیستم‌هایی که با داده‌های حساس کار می‌کنند (مانند اطلاعات مشتریان).
	\end{itemize}
	
	\textbf{مثال پیاده‌سازی:}
	فرض کنید تیم می‌خواهد سه زیرسیستم اصلی برای مدیریت سفارش، ردیابی وضعیت سفارش، و مدیریت اطلاعات مشتری طراحی کند:
	\begin{enumerate}
		\item زیرسیستم مدیریت سفارش مسئول ذخیره‌سازی و پردازش سفارش‌های مشتریان است. این زیرسیستم درخواست‌های ثبت سفارش را از طریق یک RESTful-API دریافت می‌کند و اطلاعات را در پایگاه داده ذخیره می‌کند.
		\item زیرسیستم ردیابی وضعیت سفارش از یک سیستم پیام‌محور مانند \lr{Kafka} استفاده می‌کند تا به‌صورت غیرهم‌زمان وضعیت سفارش را از زیرسیستم‌های دیگر (مانند زیرسیستم ارسال غذا) دریافت کند و آن را به‌روز کند.
		\item زیرسیستم مدیریت اطلاعات مشتری وظیفه مدیریت داده‌های مشتریان را برعهده دارد و تنها از طریق یک رابط API اطلاعات را به سایر زیرسیستم‌ها ارائه می‌دهد، بدون اینکه دسترسی مستقیم به پایگاه داده فراهم باشد.
	\end{enumerate}
	
	\textbf{نتیجه:}
	این رویکرد نه تنها استقلال عملکردی هر زیرسیستم را حفظ می‌کند، بلکه مقیاس‌پذیری و قابلیت اطمینان سیستم را نیز تضمین می‌کند. هر زیرسیستم می‌تواند جداگانه مقیاس‌دهی یا به‌روزرسانی شود بدون اینکه بر عملکرد زیرسیستم‌های دیگر تأثیر بگذارد.
\end{flushright}

