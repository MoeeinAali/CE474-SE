\subsectionaddtolist{الف}

	\textbf{۱. ریسک وابستگی به کتابخانه متن‌باز:}
	\begin{itemize}
		\item \textbf{شرح ریسک:} کتابخانه متن‌باز ممکن است دارای مشکلاتی مانند باگ‌های غیرمنتظره، توقف پشتیبانی توسط توسعه‌دهندگان، یا آسیب‌پذیری‌های امنیتی باشد که بر قابلیت اطمینان و امنیت سامانه تأثیر منفی بگذارد.
		\item \textbf{پتانسیل اثرگذاری:} در صورت وقوع این ریسک، عملکرد سامانه مختل شده یا توسعه و نگهداری پروژه با چالش مواجه خواهد شد.
		\item \textbf{راهبرد کاهش ریسک:} 
		\begin{itemize}
			\item ارزیابی دقیق کتابخانه پیش از انتخاب.
			\item استفاده از نسخه‌های پایدار و به‌روزرسانی‌های مداوم.
			\item برنامه‌ریزی برای جایگزین‌سازی کتابخانه در صورت بروز مشکلات.
		\end{itemize}
	\end{itemize}
	
	\textbf{۲. ریسک وابستگی به شرکت خارجی برای پردازش داده‌های آموزشی:}
	\begin{itemize}
		\item \textbf{شرح ریسک:} شرکت خارجی ممکن است به دلیل مشکلاتی مانند قطعی سرویس، تأخیر در پاسخ‌دهی، تغییر در سیاست‌های ارائه خدمات یا افزایش هزینه‌ها، عملکرد سامانه را تحت تأثیر قرار دهد.
		\item \textbf{پتانسیل اثرگذاری:} این ریسک می‌تواند موجب کاهش کیفیت یا قطع شدن دسترسی به داده‌های آموزشی شود که تأثیر منفی بر دقت سامانه مبتنی بر هوش مصنوعی خواهد داشت.
		\item \textbf{راهبرد کاهش ریسک:} 
		\begin{itemize}
			\item تعریف SLA (توافق‌نامه سطح خدمات) با شرکت خارجی.
			\item تهیه نسخه پشتیبان از داده‌های آموزشی.
			\item بررسی و ایجاد جایگزین‌های احتمالی برای فراخوانی API .
		\end{itemize}
	\end{itemize}
	
	این دو ریسک باید به دقت مدیریت شوند تا از تأثیرات منفی آن‌ها بر پروژه جلوگیری شود.
\subsectionaddtolist{ب}

\begin{flushright}

	
	\textbf{ریسک ۱: وابستگی به کتابخانه متن‌باز}
	\begin{itemize}
		\item \textbf{برنامه کاهش:}
		\begin{itemize}
			\item بررسی و ارزیابی مداوم کتابخانه متن‌باز، شامل آزمایش و تحلیل امنیتی پیش از استفاده.
			\item به‌کارگیری مستندات دقیق از نسخه‌های استفاده‌شده و اطمینان از سازگاری کتابخانه با نیازهای پروژه.
			\item شناسایی یک یا دو جایگزین مناسب برای کتابخانه و انجام آزمایش‌های اولیه بر روی آن‌ها.
		\end{itemize}
		\item \textbf{برنامه واکنش:}
		\begin{itemize}
			\item در صورت بروز مشکل (مانند توقف پشتیبانی)، سریعاً به یکی از کتابخانه‌های جایگزین مهاجرت کنید.
			\item تیم توسعه را آماده نگه دارید تا در صورت شناسایی باگ یا آسیب‌پذیری، تغییرات لازم را اعمال کرده یا موقتاً از بخش‌هایی از سامانه که وابسته به کتابخانه است چشم‌پوشی کنند.
		\end{itemize}
	\end{itemize}
	
	\textbf{ریسک ۲: وابستگی به شرکت خارجی برای پردازش داده‌های آموزشی}
	\begin{itemize}
		\item \textbf{برنامه کاهش:}
		\begin{itemize}
			\item عقد قرارداد با شرکت خارجی شامل توافق‌نامه سطح خدمات (SLA) که تضمین‌هایی در مورد دسترسی، کیفیت خدمات، و زمان پاسخ ارائه دهد.
			\item تهیه نسخه‌های پشتیبان از داده‌های آموزشی و ایجاد یک پایگاه داده داخلی برای ذخیره موقت داده‌ها در صورت بروز اختلال.
			\item ارزیابی و انتخاب یک یا چند ارائه‌دهنده جایگزین برای پردازش داده‌ها.
		\end{itemize}
		\item \textbf{برنامه واکنش:}
		\begin{itemize}
			\item در صورت قطعی یا خرابی خدمات، از داده‌های ذخیره‌شده در سیستم داخلی استفاده کرده و سامانه را موقتاً به حالت پردازش محلی (local-processing) تغییر دهید.
			\item در صورت تغییر سیاست‌ها یا افزایش غیرمنتظره هزینه‌ها، به ارائه‌دهنده جایگزین مهاجرت کنید یا بخشی از پردازش داده‌ها را به صورت داخلی انجام دهید.
		\end{itemize}
	\end{itemize}
	
\end{flushright}
