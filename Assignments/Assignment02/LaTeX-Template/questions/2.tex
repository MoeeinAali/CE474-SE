\subsubsection*{الف}

Impact-Mapping یک تکنیک برنامه‌ریزی در حوزه مهندسی نیازمندی‌ها است که به‌طور خاص برای تعیین و مستندسازی نحوه تحقق اهداف تجاری با استفاده از ویژگی‌ها و قابلیت‌های نرم‌افزاری طراحی شده است. هدف اصلی این تکنیک این است که درک روشنی از اهداف، رفتارها، و تأثیرات مورد نیاز برای دستیابی به این اهداف ایجاد کند.

یک Impact-Map معمولاً شامل چهار سطح مختلف است:
\begin{itemize}
    \item \textbf{هدف}: هدف تجاری یا نیاز اصلی که باید محقق شود.
    \item \textbf{مخاطب}: افرادی یا سیستم‌هایی که بر هدف تأثیر می‌گذارند.
    \item \textbf{عملکردها}: تغییراتی که در نتیجه تعاملات با سیستم باید به وجود آید.
    \item \textbf{ویژگی‌ها}: قابلیت‌ها یا ویژگی‌هایی که باید پیاده‌سازی شوند تا این تأثیرات ایجاد شود.
\end{itemize}

مزایای Impact-Mapping عبارتند از:
\begin{itemize}
    \item \textbf{تمرکز بر اهداف کسب‌وکار}: این روش کمک می‌کند تا تمرکز از ویژگی‌های فنی به اهداف تجاری انتقال یابد.
    \item \textbf{شفافیت در الزامات}: با این تکنیک می‌توان الزامات نرم‌افزاری را به‌طور واضح و بدون ابهام مستندسازی کرد.
    \item \textbf{تشخیص وابستگی‌ها}: Impact-Mapping به تیم‌ها کمک می‌کند تا وابستگی‌ها و تعاملات میان ویژگی‌ها و اهداف تجاری را شناسایی کنند.
    \item \textbf{اولویت‌بندی بهینه}: این تکنیک به اولویت‌بندی قابلیت‌ها و ویژگی‌ها بر اساس تأثیر آن‌ها بر اهداف تجاری کمک می‌کند.
\end{itemize}


\subsubsection*{ب}

روش \textbf{BDD} یک روش توسعه است که بر اساس رفتار سیستم از دیدگاه کاربر یا مشتری تمرکز دارد. BDD بر روی توصیف سناریوهایی متمرکز است که نحوه تعامل سیستم با کاربران و سایر بازیگران را مشخص می‌کند. این روش با استفاده از زبان طبیعی برای نوشتن تست‌های رفتاری کاربرد دارد.

رابطه بین Impact-Map و BDD در این است که هر دو بر اساس اهداف تجاری و نیازهای کاربران تمرکز دارند. Impact-Map از طریق شناسایی اهداف و تأثیرات مختلف، به تعیین رفتارهای مورد نیاز برای دستیابی به اهداف کمک می‌کند. سپس، این رفتارها می‌توانند به سناریوهای BDD تبدیل شوند تا تست‌های رفتاری دقیق‌تری برای پیاده‌سازی ویژگی‌ها و قابلیت‌ها ایجاد کنند.

\begin{itemize}
    \item \textbf{تطابق اهداف تجاری با سناریوهای BDD}: Impact-Map به شناسایی اهداف تجاری کمک می‌کند که می‌تواند به سناریوهای BDD تبدیل شود. این سناریوها نشان‌دهنده نحوه تعامل کاربران با سیستم برای رسیدن به این اهداف هستند.
    \item \textbf{ارتباط دقیق با رفتار سیستم}: Impact-Map رفتارهای مورد نیاز سیستم را بر اساس نیازهای تجاری مشخص می‌کند که با استفاده از BDD به زبان قابل فهم برای تمامی اعضای تیم مستند می‌شود.
\end{itemize}

\pagebreak
\subsubsection*{ج}

برای ترکیب Impact-Map با روش BDD ، می‌توان مراحل زیر را دنبال کرد:

\begin{enumerate}
    \item \textbf{شناسایی اهداف تجاری}: ابتدا باید اهداف تجاری پروژه را مشخص کرد. این اهداف در Impact-Map تعریف می‌شوند.
    \item \textbf{شناسایی بازیگران و تأثیرات}: در این مرحله، بازیگران مختلف که می‌توانند بر اهداف تجاری تأثیر بگذارند، شناسایی می‌شوند و تأثیراتی که باید در نتیجه این تعاملات اتفاق بیفتد، مشخص می‌شود.
    \item \textbf{ترجمه تأثیرات به سناریوهای BDD}: تأثیرات شناسایی‌شده می‌توانند به سناریوهای BDD تبدیل شوند. این سناریوها باید به‌طور خاص رفتار سیستم را در مواجهه با هر بازیگر توصیف کنند.
    \item \textbf{اولویت‌بندی سناریوها}: از آنجا که Impact-Mapping به تعیین اولویت‌ها کمک می‌کند، سناریوهای BDD نیز می‌توانند بر اساس اولویت‌های تجاری تعیین‌شده در Impact-Map اولویت‌بندی شوند.
    \item \textbf{اجرای تست‌های رفتاری}: در نهایت، سناریوهای BDD به تست‌های رفتاری تبدیل می‌شوند که باید پیاده‌سازی شده و به‌طور مداوم آزمایش شوند.
\end{enumerate}
