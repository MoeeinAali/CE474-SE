\subsectionaddtolist{بخش اول}

\subsubsection*{الف}


نیازمندی‌های غیرعملکردی (NFRs) به مجموعه‌ای از ویژگی‌ها و معیارهایی اشاره دارند که نحوه‌ی عملکرد یک سیستم را تعریف می‌کنند. این نیازمندی‌ها به جای تعیین اینکه سیستم چه کارهایی باید انجام دهد، بر روی کیفیت و نحوه‌ی انجام آن‌ها تمرکز دارند. نیازمندی‌های غیرعملکردی شامل ویژگی‌هایی مانند عملکرد، امنیت، قابلیت استفاده، و قابلیت اطمینان هستند که در موفقیت کلی یک سیستم نرم‌افزاری تأثیرگذارند.

\textbf{تفاوت بین نیازمندی‌های عملکردی و غیرعملکردی:}
\begin{itemize}
    \item \textbf{نیازمندی‌های عملکردی:} این نیازمندی‌ها مشخص می‌کنند که سیستم باید چه وظایف یا عملکردهایی را انجام دهد. یعنی چه کاری باید انجام شود.
    \item \textbf{نیازمندی‌های غیرعملکردی:} این نیازمندی‌ها مشخص می‌کنند که سیستم چگونه باید این وظایف را انجام دهد. یعنی چطور کاری باید انجام شود.
\end{itemize}

\textbf{مثال‌ها:}
\begin{itemize}
    \item \textbf{نیازمندی عملکردی:} سیستم باید به کاربران اجازه دهد که با ایمیل و رمز عبور وارد شوند.
    \item \textbf{نیازمندی غیرعملکردی:} سیستم باید قادر باشد که در کمتر از 2 ثانیه وارد شود.
    \item \textbf{نیازمندی عملکردی:} سیستم باید به کاربران اجازه دهد تا سوابق پزشکی خود را به روز کنند.
    \item \textbf{نیازمندی غیرعملکردی:} سیستم باید اطمینان حاصل کند که سوابق پزشکی به طور امن و بدون از دست رفتن داده‌ها به روز می‌شود.
    \item \textbf{نیازمندی عملکردی:} سیستم باید به کاربران اجازه دهد پیام‌هایی به پزشکان ارسال کنند.
    \item \textbf{نیازمندی غیرعملکردی:} سیستم باید قابلیت اطمینان 99\% برای خدمات پیام‌رسانی را فراهم کند.
\end{itemize}


\subsubsection*{ب}


نیازمندی‌های غیرعملکردی مانند عملکرد، امنیت و مقیاس‌پذیری نقشی حیاتی در موفقیت پروژه دارند زیرا به طور مستقیم بر توانایی سیستم در برآورده کردن انتظارات کاربران و بقای بلندمدت آن تأثیر می‌گذارند.

\begin{itemize}
    \item \textbf{عملکرد:} نیازمندی‌های غیرعملکردی اطمینان می‌دهند که سیستم به زمان‌های پاسخ‌دهی و توان پردازش مناسب برسد. عملکرد ضعیف می‌تواند منجر به ناراحتی کاربران، زمان‌های تأخیر و عدم کارایی سیستم شود. برای مثال، تأخیر در بارگذاری سوابق پزشکی در یک سیستم بهداشتی می‌تواند منجر به تأخیر در مشاوره‌های پزشکی یا تجویز دارو شود.
    \item \textbf{امنیت:} نیازمندی‌های امنیتی برای حفاظت از داده‌های حساس در برابر دسترسی‌های غیرمجاز، حملات سایبری و نقض حریم خصوصی ضروری هستند. امنیت قوی می‌تواند از نشت داده‌ها جلوگیری کرده و اعتماد کاربران را حفظ کند.
    \item \textbf{مقیاس‌پذیری:} مقیاس‌پذیری برای سیستم‌هایی که انتظار دارند حجم بیشتری از کاربران یا داده‌ها را در آینده مدیریت کنند، حیاتی است. اگر سیستم نتواند به درستی مقیاس‌پذیر شود، ممکن است با افزایش تعداد کاربران یا حجم داده‌ها دچار افت عملکرد یا حتی خرابی شود. مقیاس‌پذیری این اطمینان را می‌دهد که سیستم به طور یکپارچه و بدون مشکل گسترش یابد.
\end{itemize}



\subsectionaddtolist{بخش دوم}
\subsubsection*{الف}

با توجه به مشکلات گزارش‌شده از سوی کاربران، نیازمندی‌های غیرعملکردی زیر نقض شده‌اند:

\begin{itemize}
    \item \textbf{نقض عملکرد:} تأخیر در بارگذاری سوابق پزشکی، به ویژه تصاویر و ویدیوها، در حین کنفرانس‌های ویدیویی نشان‌دهنده نقض عملکرد است. نیازمندی‌های عملکردی ایجاب می‌کنند که بارگذاری داده‌ها در مدت زمان معقول انجام شود. عدم رعایت این نیازمندی باعث تأخیر قابل توجهی در عملکرد سیستم می‌شود.
    \item \textbf{نقض قابلیت دسترسی پذیری:} نیاز به قطع و وصل مجدد اتصال زمانی که کاربران سوابق پزشکی خود را مشاهده یا به‌روزرسانی می‌کنند، نشان‌دهنده نقض در دسترسی سیستم است. نیازمندی‌های قابلیت دسترسی ایجاب می‌کنند که سیستم بدون قطع شدن یا وقفه مکرر در دسترس باشد.
    \item \textbf{نقض امنیت:} دسترسی تصادفی برخی از کاربران به سوابق پزشکی بیماران دیگر نقض نیازمندی‌های امنیتی است. این نیازمندی‌ها ایجاب می‌کنند که داده‌های بیمار باید به‌طور کامل از هم جدا شده و از دسترسی غیرمجاز محافظت شوند.
\end{itemize}

\pagebreak
\subsubsection*{ب}

برای ارزیابی اینکه نیازمندی‌های غیرعملکردی به درستی در سیستم پیاده‌سازی شده‌اند، می‌توان از شاخص‌های زیر استفاده کرد:

\begin{itemize}
    \item \textbf{عملکرد:} 
    \begin{itemize}
        \item \textbf{شاخص:} زمان پاسخ‌دهی برای بارگذاری سوابق و کنفرانس‌های ویدیویی.
        \item \textbf{چگونه ارزیابی می‌کند:} این شاخص زمان لازم برای بارگذاری فایل‌ها و راه‌اندازی کنفرانس ویدیویی را اندازه‌گیری می‌کند و اطمینان می‌دهد که زمان‌های تأخیر برای بارگذاری فایل‌ها از حد مجاز (مثلاً 5 ثانیه) بیشتر نشود.
    \end{itemize}
    \item \textbf{قابلیت دسترسی:}
    \begin{itemize}
        \item \textbf{شاخص:} درصد زمان در دسترس بودن سیستم.
        \item \textbf{چگونه ارزیابی می‌کند:} این شاخص زمان در دسترس بودن سیستم را بر اساس درصد زمانی که سیستم بدون وقفه یا خرابی کار می‌کند، اندازه‌گیری می‌کند. هدف این است که در دسترس بودن سیستم 99\% باشد.
    \end{itemize}
    \item \textbf{امنیت:}
    \begin{itemize}
        \item \textbf{شاخص:} تعداد حوادث دسترسی غیرمجاز.
        \item \textbf{چگونه ارزیابی می‌کند:} این شاخص تعداد دفعاتی که کاربران غیرمجاز به داده‌ها دسترسی پیدا کرده‌اند را اندازه‌گیری می‌کند. باید تعداد این حوادث برابر صفر باشد تا امنیت سیستم تضمین شود.
    \end{itemize}
\end{itemize}

\subsubsection*{ج}

چندین عامل ممکن است باعث نقض نیازمندی‌های غیرعملکردی شوند:

\begin{itemize}
    \item \textbf{عملکرد:} ممکن است سیستم برای مدیریت فایل‌های بزرگ (تصاویر و ویدیوها) بهینه‌سازی نشده باشد که باعث تأخیر در بارگذاری می‌شود. همچنین ممکن است سیستم تحت فشار تعداد زیادی از درخواست‌ها قرار گیرد و منابع به درستی تخصیص نیابند.
    \item \textbf{قابلیت دسترسی:} قطع شدن سرویس به دلیل مشکلات سخت‌افزاری، ناتوانی در مقیاس‌پذیری یا عدم پشتیبانی از زیرساخت مناسب می‌تواند باعث بروز اختلال در دسترسی سیستم شود.
    \item \textbf{امنیت:} عدم پیاده‌سازی مکانیسم‌های امنیتی قوی، مانند رمزنگاری مناسب یا تأیید هویت، می‌تواند باعث نقض امنیت و دسترسی غیرمجاز به داده‌ها شود.
\end{itemize}


\subsubsection*{د}

برای بهبود نیازمندی‌های غیرعملکردی سیستم، چندین راه‌حل فنی و مدیریتی وجود دارد که می‌توانند به بهبود عملکرد، امنیت، و مقیاس‌پذیری سیستم کمک کنند. در این بخش، به شرح راه‌حل‌های فنی و مدیریتی می‌پردازیم:

\subsubsection*{1. بهبود عملکرد}

\begin{itemize}
    \item \textbf{راه‌حل فنی: استفاده از فشرده‌سازی داده‌ها}
    \begin{itemize}
        \item \textit{توضیح:} استفاده از الگوریتم‌های فشرده‌سازی پیشرفته مانند JPEG برای تصاویر و H.264 برای ویدیوها می‌تواند حجم داده‌ها را کاهش دهد و زمان بارگذاری را به حداقل برساند.
        \item \textit{تأثیر:} این راه‌حل باعث کاهش زمان بارگذاری و انتقال داده‌ها و در نتیجه بهبود عملکرد سیستم خواهد شد.
    \end{itemize}
    
    \item \textbf{راه‌حل فنی: استفاده از بارگذاری همزمان}
    \begin{itemize}
        \item \textit{توضیح:} بارگذاری همزمان تصاویر و ویدیوها از طریق استفاده از چندین ترد می‌تواند سرعت بارگذاری را بهبود دهد.
        \item \textit{تأثیر:} با این روش، زمان بارگذاری کلی کاهش یافته و کارایی سیستم بهتر خواهد شد.
    \end{itemize}
\end{itemize}

\subsubsection*{2. بهبود قابلیت دسترسی پذیری}

\begin{itemize}
    \item \textbf{راه‌حل فنی: استفاده از سیستم‌های Caching}
    \begin{itemize}
        \item \textit{توضیح:} برای بهبود دسترسی به داده‌ها، می‌توان از سیستم‌های کشینگ مانند Redis یا سیستم‌های توزیع‌شده استفاده کرد که داده‌ها را در نزدیکی کاربر ذخیره می‌کنند.
        \item \textit{تأثیر:} این روش باعث کاهش بار بر روی سرور و افزایش سرعت دسترسی به اطلاعات خواهد شد.
    \end{itemize}
    
    \item \textbf{راه‌حل مدیریتی: پیاده‌سازی نظارت مستمر}
    \begin{itemize}
        \item \textit{توضیح:} با استفاده از ابزارهای نظارت مانند Prometheus یا Grafana ، عملکرد سیستم و وضعیت منابع به‌طور مداوم بررسی و بهینه‌سازی می‌شود.
        \item \textit{تأثیر:} این روش باعث جلوگیری از خرابی‌های غیرمنتظره و بهبود دسترسی به سیستم خواهد شد.
    \end{itemize}
\end{itemize}

\subsubsection*{3. بهبود امنیت}

\begin{itemize}
    \item \textbf{راه‌حل فنی: پیاده‌سازی رمزنگاری End-to-End}
    \begin{itemize}
        \item \textit{توضیح:} رمزنگاری End-to-End باعث می‌شود که تنها فرستنده و گیرنده قادر به مشاهده داده‌ها باشند. این روش از اطلاعات حساس کاربران در برابر حملات محافظت می‌کند.
        \item \textit{تأثیر:} این راه‌حل امنیت داده‌ها را به‌شدت افزایش می‌دهد و از دسترسی غیرمجاز جلوگیری می‌کند.
    \end{itemize}
    
    \item \textbf{راه‌حل فنی: پیاده‌سازی MFA}
    \begin{itemize}
        \item \textit{توضیح:} با استفاده از احراز هویت چندعاملی، دسترسی به سیستم تنها با وارد کردن کد ارسال‌شده به گوشی همراه یا ایمیل ممکن می‌شود.
        \item \textit{تأثیر:} این روش لایه امنیتی اضافی ایجاد می‌کند و خطر دسترسی غیرمجاز به سیستم را به حداقل می‌رساند.
    \end{itemize}
\end{itemize}

\subsubsection*{4. بهبود مقیاس‌پذیری}

\begin{itemize}
    \item \textbf{راه‌حل فنی: استفاده از معماری میکروسرویس}
    \begin{itemize}
        \item \textit{توضیح:} با استفاده از معماری میکروسرویس‌ها، سیستم به بخش‌های کوچکتری تقسیم می‌شود که می‌توانند به‌طور مستقل مقیاس‌پذیر باشند. این معماری باعث می‌شود که سیستم به‌راحتی در صورت افزایش تعداد کاربران مقیاس‌پذیر شود.
        \item \textit{تأثیر:} این راه‌حل مقیاس‌پذیری سیستم را بهبود می‌دهد و امکان گسترش سیستم بدون اختلال در بخش‌های دیگر فراهم می‌شود.
    \end{itemize}
    
    \item \textbf{راه‌حل مدیریتی: توسعه برنامه‌های مقیاس‌پذیر از ابتدا}
    \begin{itemize}
        \item \textit{توضیح:} تیم مدیریت باید از ابتدا مقیاس‌پذیری را به عنوان یک اولویت در طراحی سیستم در نظر بگیرد. این شامل انتخاب زیرساخت‌های مقیاس‌پذیر و استفاده از الگوریتم‌های مناسب است.
        \item \textit{تأثیر:} این روش باعث می‌شود که سیستم در برابر بارهای بالا و افزایش تعداد کاربران مقاوم باشد و قابلیت گسترش آن به‌راحتی امکان‌پذیر شود.
    \end{itemize}
\end{itemize}


\subsectionaddtolist{بخش سوم}


\subsubsection*{الف}


در سیستم‌هایی مانند سیستم کنترل ترافیک هوایی که به دقت بالا و امنیت قوی نیاز دارند، همیشه تعارضاتی بین الزامات مختلف وجود دارد. برای مثال، در این سیستم، نیاز به امنیت بالا که شامل لایه‌های مختلف امنیتی و رمزنگاری داده‌هاست، باعث افزایش زمان پردازش داده‌ها می‌شود. این در حالی است که دقت و سرعت در انجام عملیات برای اطمینان از مدیریت بی‌وقفه و بدون تاخیر ترافیک هوایی بسیار حیاتی است. 

برای مدیریت این "توافقات" باید اولویت‌های پروژه مشخص شود. ممکن است که برای حفظ امنیت داده‌ها، باید میزان پردازش بیشتری انجام گیرد که موجب تاخیر در پردازش‌ها شود. از طرف دیگر، این افزایش تاخیر ممکن است تاثیرات منفی بر عملکرد سیستم در محیط‌های پر ترافیک داشته باشد. به همین دلیل، نیاز به یک روش تصمیم‌گیری مناسب برای ایجاد تعادل بین این الزامات وجود دارد. برای مدیریت این چالش می‌توان از روش‌های زیر استفاده کرد:
\begin{itemize}
    \item \textbf{محدودسازی پیچیدگی‌ها}: در ابتدای طراحی باید پیچیدگی‌های غیر ضروری در الگوریتم‌ها و لایه‌های امنیتی حذف شود تا از پیچیدگی‌های اضافی جلوگیری شود.
    \item \textbf{تخصیص منابع بهینه}: تخصیص مناسب منابع برای هر الزامات عملکردی و غیرعملکردی (مانند امنیت) می‌تواند کمک کند تا تعادل میان امنیت و سرعت پردازش برقرار شود.
    \item \textbf{ارزیابی و تست مداوم}: اجرای تست‌های مختلف برای ارزیابی اثرات متقابل بین الزامات مختلف و یافتن بهترین نقطه تعادل برای سیستم.
\end{itemize}



\subsubsection*{ب}

در طول چرخه عمر پروژه، ممکن است الزامات جدیدی به سیستم افزوده شوند. برای مثال، با توجه به درخواست‌های جدید برای استفاده از سیستم در فرودگاه‌های بین‌المللی با ترافیک بالاتر، نیاز به مقیاس‌پذیری و بهبود سیستم خواهیم داشت. در این صورت، امکان بروز تغییرات زیاد در معماری سیستم، الگوریتم‌ها و لایه‌های امنیتی وجود دارد که می‌تواند اثرات منفی بر پایداری و عملکرد سیستم بگذارد. این افزایش الزامات به چالش‌هایی مانند تغییرات در ساختار داده‌ها، هزینه‌های اضافی، و لزوم اطمینان از کارکرد بی‌وقفه سیستم (که جزو الزامات حیاتی است) منجر می‌شود.

برای مدیریت این چالش می‌توان از روش‌ها و ابزارهای زیر استفاده کرد:
\begin{itemize}
    \item \textbf{مدیریت تغییرات}: ایجاد یک فرآیند استاندارد برای پذیرش، ارزیابی و پیاده‌سازی تغییرات که شامل بررسی دقیق تأثیرات تغییرات جدید بر سیستم باشد.
    \item \textbf{استفاده از ابزارهای نظارتی و مدیریتی}: به کارگیری ابزارهای نظارتی برای پایش عملکرد سیستم و شناسایی مشکلات و نیازهای جدید.
    \item \textbf{مقیاس‌پذیری و انعطاف‌پذیری در طراحی}: استفاده از طراحی‌های مقیاس‌پذیر که بتوانند با افزایش بار و تغییرات مختلف سازگار شوند.
    \item \textbf{تست و ارزیابی مداوم}: اجرای تست‌های دوره‌ای برای ارزیابی قابلیت مقیاس‌پذیری، امنیت و پایداری سیستم در برابر تغییرات و افزایش بار ترافیکی.
\end{itemize}