\subsectionaddtolist{الف}
یکپارچگی در فرآیند تست نرم‌افزار به معنای بررسی نحوه تعامل و همکاری میان ماژول‌ها و اجزای مختلف یک سامانه نرم‌افزاری است. این تست اطمینان حاصل می‌کند که اجزا به درستی با یکدیگر ارتباط دارند و داده‌ها بین آن‌ها به‌صورت صحیح منتقل می‌شوند. 

اهمیت تست‌های یکپارچگی برای سامانه‌های پیچیده به دلایل زیر است:
\begin{itemize}
    \item تضمین سازگاری بین ماژول‌های مختلف.
    \item کشف خطاهای ناشی از تعامل اجزا که در تست‌های واحد قابل تشخیص نیستند.
    \item افزایش قابلیت اطمینان سامانه و کاهش مشکلات احتمالی در محیط عملیاتی.
\end{itemize}

تفاوت‌های کلیدی بین تست‌های واحد و تست‌های یکپارچگی:
\begin{itemize}
    \item تست واحد بر روی یک ماژول یا تابع مجزا تمرکز دارد، در حالی که تست یکپارچگی نحوه تعامل چندین ماژول را بررسی می‌کند.
    \item تست واحد معمولاً توسط توسعه‌دهندگان انجام می‌شود، اما تست یکپارچگی بیشتر توسط تیم تست انجام می‌شود.
    \item در تست واحد، از شبیه‌سازی (Mock) برای بررسی عملکرد مستقل یک ماژول استفاده می‌شود، اما در تست یکپارچگی، سیستم در محیط واقعی‌تری تست می‌شود.
\end{itemize}

\subsectionaddtolist{ب}
دو استراتژی مهم تست یکپارچگی عبارتند از:
\begin{itemize}
    \item {ادغام بیگ‌بنگ}: در این روش، تمام ماژول‌های سامانه به‌طور همزمان با یکدیگر ادغام و سپس تست می‌شوند. این روش برای سامانه‌های پیچیده ریسک بالایی دارد، زیرا خطاهای متعددی به‌طور همزمان پدیدار می‌شوند و یافتن منبع اصلی مشکل دشوار است.
    \item {ادغام افزایشی}: در این روش، ماژول‌ها به‌صورت تدریجی و مرحله‌به‌مرحله ادغام و تست می‌شوند. این روش شامل دو رویکرد اصلی است:
    \begin{itemize}
        \item {ادغام از بالا به پایین}: ابتدا ماژول‌های سطح بالا تست می‌شوند و سپس ماژول‌های سطح پایین اضافه می‌شوند.
        \item {ادغام از پایین به بالا}: ابتدا ماژول‌های سطح پایین ادغام و تست شده و سپس به سطح بالاتر گسترش می‌یابند.
    \end{itemize}
\end{itemize}

برای تست سامانه بانک اطلاعاتی، استراتژی {ادغام افزایشی} مناسب‌تر است، زیرا:
\begin{itemize}
    \item کاهش ریسک خطاهای پیچیده و امکان شناسایی زودهنگام مشکلات.
    \item امکان انجام تست‌های مرحله‌به‌مرحله برای اطمینان از عملکرد صحیح هر بخش.
    \item ساده‌تر بودن اشکال‌زدایی نسبت به روش بیگ‌بنگ.
\end{itemize}

\subsectionaddtolist{ج}
\textbf{سناریوی اول:}
\begin{itemize}
    \item {مرحله 1}: کاربر اطلاعات ورود را در سامانه ورود وارد می‌کند.
    \item {مرحله 2}: سامانه ورود، اطلاعات کاربر را به سامانه مدیریت داده‌ها ارسال می‌کند.
    \item {مرحله 3}: سامانه مدیریت داده‌ها بررسی می‌کند که آیا اطلاعات کاربر معتبر است یا خیر.
    \item {مرحله 4}: در صورت تأیید، مجوز ورود داده شده و اطلاعات به کاربر نمایش داده می‌شود.
\end{itemize}

{خطاهای محتمل:}
\begin{itemize}
    \item نامعتبر بودن اطلاعات ورود (ورود نادرست رمز عبور).
    \item اختلال در ارتباط میان سامانه ورود و سامانه مدیریت داده‌ها.
    \item ارسال اطلاعات نادرست از سامانه ورود به سامانه مدیریت داده‌ها.
\end{itemize}

\textbf{سناریوی دوم:}
\begin{itemize}
    \item {مرحله 1}: یک تراکنش مالی توسط کاربر انجام می‌شود.
    \item {مرحله 2}: سامانه مدیریت تراکنش‌ها جزئیات تراکنش را پردازش و ذخیره می‌کند.
    \item {مرحله 3}: سامانه مدیریت تراکنش‌ها داده‌های مرتبط را به سامانه گزارش‌دهی ارسال می‌کند.
    \item {مرحله 4}: سامانه گزارش‌دهی داده‌ها را دریافت کرده و یک گزارش مالی تولید می‌کند.
\end{itemize}

{خطاهای محتمل:}
\begin{itemize}
    \item تأخیر یا عدم ارسال اطلاعات از سامانه مدیریت تراکنش‌ها به سامانه گزارش‌دهی.
    \item نمایش نادرست اطلاعات تراکنش در گزارش‌ها.
    \item ناهماهنگی در پردازش داده‌ها و ایجاد گزارش‌های ناقص.
\end{itemize}


