\subsectionaddtolist{الف}
کیفیت نرم‌افزار به معنای میزان برآورده شدن نیازهای عملکردی و غیرعملکردی سیستم است. دستیابی به کیفیت بالا در مهندسی نرم‌افزار حیاتی است زیرا مستقیماً بر قابلیت اطمینان، نگهداشت‌پذیری، و رضایت کاربران تأثیر می‌گذارد. کیفیت پایین نرم‌افزار می‌تواند منجر به نقص‌های پرهزینه، کاهش بهره‌وری، و نارضایتی کاربران شود.

ابعاد کیفیت مطرح شده توسط گاروین عبارتند از:

\begin{itemize}
    \item {عملکرد}(Performance) : میزان توانایی سیستم در انجام وظایف موردنظر. مثال: سامانه مدیریت بانکداری الکترونیکی که تراکنش‌ها را به‌سرعت پردازش می‌کند.
    \item {ویژگی‌ها}(Features) : قابلیت‌های اضافی که فراتر از نیازهای اولیه هستند. مثال: اپلیکیشن پیام‌رسانی که علاوه بر ارسال متن، امکان تماس ویدیویی نیز دارد.
    \item {قابلیت اطمینان}(Reliability) : میزان عملکرد پایدار و بدون خرابی سیستم در طول زمان. مثال: سیستم کنترلی هواپیما که باید همواره بدون نقص کار کند.
    \item {مطابقت}(Conformance) : میزان تطابق با استانداردها و مشخصات موردنظر. مثال: نرم‌افزار حسابداری که مطابق با استانداردهای مالیاتی کشور توسعه یافته است.
    \item {دوام}(Durability) : پایداری و طول عمر نرم‌افزار. مثال: سیستم عامل که به مدت طولانی قابل استفاده باشد.
    \item {قابلیت استفاده}(Usability) : میزان سهولت استفاده توسط کاربران. مثال: نرم‌افزار طراحی گرافیکی با رابط کاربری ساده و شهودی.
    \item {قابلیت سرویس‌دهی}(Serviceability) : سهولت نگهداری و اصلاح نرم‌افزار. مثال: سامانه مدیریت محتوا (CMS) که به‌روزرسانی‌های آسان دارد.

\end{itemize}

\subsectionaddtolist{ب}
متدولوژی‌های آزمون نرم‌افزار را می‌توان به دو دسته‌ی آزمون مرسوم و آزمون شی‌گرا تقسیم کرد.

\begin{itemize}
    \item {سطوح آزمون}: در آزمون مرسوم، تست واحدها (Unit-Testing) به‌صورت تابعی انجام می‌شود، اما در آزمون شی‌گرا، تمرکز بر کلاس‌ها و اشیا است.
    \item {مورد آزمون}: در آزمون مرسوم، تست بر روی ماژول‌های منفرد انجام می‌شود، اما در آزمون شی‌گرا تعامل بین اشیا و پیام‌ها بررسی می‌شود.
    \item {ابزارها}: ابزارهای آزمون سنتی شامل JUnit و NUnit می‌باشند، در حالی که آزمون شی‌گرا از ابزارهایی مانند JMock و Mockito بهره می‌برد.
    \item {مزایا}: آزمون شی‌گرا مقیاس‌پذیری بیشتری دارد و قابلیت نگهداشت بالاتری را ارائه می‌دهد.
\end{itemize}

\subsectionaddtolist{ج}
{راستی‌آزمایی (Verification)} و {صحت‌سنجی (Validation)} دو مرحله‌ی مهم در آزمون نرم‌افزار هستند:

\begin{itemize}
    \item {راستی‌آزمایی}: فرآیند بررسی این که آیا نرم‌افزار مطابق با مشخصات طراحی شده است یا خیر. مثال: بررسی مستندات و اجرای آزمون‌های واحد.
    \item {صحت‌سنجی}: بررسی اینکه آیا نرم‌افزار نیازهای کاربر را برآورده می‌کند یا خیر. مثال: انجام آزمون‌های پذیرش توسط کاربران.
\end{itemize}

هر دو پروسه برای تضمین کیفیت بالا ضروری هستند، زیرا راستی‌آزمایی از تطابق با طراحی و صحت‌سنجی از تطابق با نیازهای واقعی اطمینان حاصل می‌کند.

\subsectionaddtolist{د}
چندین اقدام می‌تواند تأثیر {آزمون رگرسیون} را بهبود ببخشد:

\begin{itemize}
    \item استفاده از {آزمون‌های خودکار} برای کاهش هزینه و زمان آزمون.
    \item انتخاب {مجموعه‌ی بهینه آزمون‌ها} برای اجرای مجدد، به جای اجرای کل مجموعه.
    \item استفاده از {آزمون‌های افزایشی} به جای اجرای مجدد کل سیستم.
    \item ترکیب آزمون رگرسیون با {تحلیل تأثیر تغییرات} برای تمرکز بر قسمت‌های حساس نرم‌افزار.
\end{itemize}
